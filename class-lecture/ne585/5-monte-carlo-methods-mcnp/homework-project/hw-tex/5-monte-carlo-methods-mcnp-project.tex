%%%%%
%@TheDoctorRAB
%%%%%
%
%REFERENCES
%
%neup.bst - numbered citations in order of appearance, short author list with et al in reference section
%nsf.bst - numbered citations in order of appearance, full author list in references section
%standard.bst - citations with author last name with et al for more than 2 authors; full author list in references section
%ans.bst is for ANS only. 
%
%author = {Lastname, Firstname and Lastname, Firstname and Lastname, Firstname} for all bst formats
%bst renders the author list itself
%
%author = {{Nuclear Regulatory Commission}} if the author is an organization, institution, etc., and not people
%
%title = {{}} for all
%
%for all - use \citep{-} - [1] or (Borrelli, 2021) in the text
%standard.bst \cite{-} - Borrelli (2021) in the text
%standard.bst lists references alphabetically
%the rest list numerically
%%%%%


%%%%% general 
\documentclass[11pt,a4paper]{article}
\usepackage[lmargin=1in,rmargin=1in,tmargin=1in,bmargin=1in]{geometry}
\usepackage[pagewise]{lineno} %line numbering
\usepackage{setspace}
\usepackage{ulem} %strikethrough - do not \sout{\cite{}}
\usepackage[pdftex,dvipsnames]{xcolor,colortbl} %change font color
\usepackage{graphicx}
\usepackage{filecontents}
\usepackage{tablefootnote}
\usepackage{footnotehyper}
\usepackage{float}
\usepackage{mypythonhighlight,verbatim}
%\usepackage{subfig}
\usepackage[yyyymmdd]{datetime} %date format
\renewcommand{\dateseparator}{.}
\graphicspath{{../img/}} %path to graphics
\setcounter{secnumdepth}{5} %set subsection to nth level
\usepackage{needspace}
\usepackage[stable,hang,flushmargin]{footmisc} %footnotes in section titles and no indent; standard.bst
\usepackage[inline]{enumitem}
\usepackage{boldline}
\usepackage{makecell}
\usepackage{booktabs}
\usepackage{amssymb}
\usepackage{gensymb}
\usepackage{amsmath,nicefrac}
\usepackage{physics}
\usepackage{lscape}
\usepackage{array}
\usepackage{chngcntr}
\usepackage{hyperref}
\hypersetup{colorlinks,linkcolor=black,citecolor=black,urlcolor=blue} 
\usepackage{sectsty}
\usepackage{textcomp}
\usepackage{lastpage}
\usepackage{xargs} %for \newcommandx
\usepackage[colorinlistoftodos,prependcaption,textsize=tiny]{todonotes} %makes colored boxes for commenting
\usepackage{soul}
\usepackage{color}
\usepackage{marginnote}
\usepackage[figure,table]{totalcount}
\usepackage[capitalise]{cleveref}
\usepackage{nameref}
\usepackage{parskip}
%%%%%


%%%%% tikz
\usepackage{pgf}
\usepackage{tikz} % required for drawing custom shapes
\usetikzlibrary{shapes,arrows,automata,trees}
%%%%%


%%%%% fonts
\usepackage{times}
%arial - uncomment next two lines
%\usepackage{helvet}
%\renewcommand{\familydefault}{\sfdefault}
%%%%%


%%%%% references
%\usepackage[round,semicolon]{natbib} %for (Borrelli 2021; Clooney 2019) - standard.bst 
\usepackage[numbers,sort&compress]{natbib} %for [1-3] - nsf.bst, neup.bst
\setlength{\bibsep}{7pt} %sets space between references
%\renewcommand{\bibsection}{} %suppresses large 'references' heading
%\renewcommand\bibpreamble{\vspace{\baselineskip}} %sets spacing after heading if not using default references heading
%%%%%


%%%%% tables and figures
\usepackage{longtable} %need to put label at top under caption then \\ - use spacing
\usepackage{tablefootnote}
\usepackage{tabularx}
\usepackage{multirow}
\usepackage{tabto} %general tabbed spacing
\usepackage{pdfpages}
\usepackage{wrapfig} %wraps figures around text
\setlength{\intextsep}{0.00mm}
\setlength{\columnsep}{1.00mm}
\usepackage[singlelinecheck=false,labelfont=bf]{caption}
\usepackage{subcaption}
\captionsetup[table]{justification=justified,skip=5pt,labelformat={default},labelsep=period,name={Table}} %sets a space after table caption
\captionsetup[figure]{justification=justified,skip=5pt,labelformat={default},labelsep=period,name={Figure}} %sets space above caption, 'figure' format
\captionsetup[wrapfigure]{justification=centering,aboveskip=0pt,belowskip=0pt,labelformat={default},labelsep=period,name={Fig.}} %sets space above caption, 'figure' format
\captionsetup[wraptable]{justification=centering,aboveskip=0pt,belowskip=0pt,labelformat={default},labelsep=period,name={Table}} %sets space above caption, 'figure' format
%%%%%


%%%%% watermark
%\usepackage[firstpage,vpos=0.63\paperheight]{draftwatermark}
%\SetWatermarkText{\shortstack{CONFIDENTIAL\\do not distribute}}
%\SetWatermarkScale{0.20}
%%%%%


%%%%% cross referencing files
%\usepackage{xr} %for revisions - will cross reference from one file to here
%\externaldocument{../review/PNUCENE-D-22-00841-Reviewer1} %aux file needed
%\externaldocument{../review/PNUCENE-D-22-00841-Reviewer2} %aux file needed
%see reviwer tex file for compiling instructions
%%%%%


%%%%% toc and glossaries
\usepackage[toc,title]{appendix}
\usepackage[acronym,nomain,nonumberlist]{glossaries}
\makenoidxglossaries
\usepackage{titlesec,titletoc}
%\renewcommand{\thepart}{ARTICLE \Roman{part}} %puts the label into the command so \thelabel will carry through
%\renewcommand{\thesection}{\arabic{section}} %puts the label into the command so \thelabel will carry through
%\titleformat{\part}{\normalfont\Large\bfseries\filcenter}{\thepart}{}{}[]
%\titlespacing*\part{0pt}{0.95\baselineskip}{0.75\baselineskip}
%\titleformat{\section}{\normalfont\large\bfseries\filcenter}{\thesection}{}{}[]
%\titlespacing*\section{0pt}{0.65\baselineskip}{0.35\baselineskip}
%\titleformat{\subsection}[runin]{\normalfont\normalsize\bfseries}{\thesubsection}{-1em}{}[.]
%\titlespacing*\subsection{0pt}{0.50\baselineskip}{0.25\baselineskip}
%\titleformat{\paragraph}[runin]{\normalfont\normalsize\bfseries\itshape}{\theparagraph}{-1em}{}[.]
%\titlespacing*\paragraph{0pt}{0.45\baselineskip}{0.25\baselineskip}
%\titleformat{\subparagraph}[runin]{\normalfont\normalsize\itshape}{\thesubparagraph}{-1em}{}[.]
%\titlespacing*\subparagraph{0pt}{0.40\baselineskip}{0.25\baselineskip}
\titleformat{\paragraph}[hang]{\normalfont\normalsize\bfseries}{\theparagraph}{5pt}{}[]
\titlespacing*\paragraph{0pt}{0.50\baselineskip}{0.25\baselineskip}
\titleformat{\subparagraph}[runin]{\normalfont\normalsize\itshape}{\thesubparagraph}{-1em}{}[.]
\titlespacing*\subparagraph{0pt}{0.40\baselineskip}{0.20\baselineskip}
%%%%%


%%%%% editing
\newcommand{\edit}[1]{\textcolor{blue}{#1}} %shortcut for changing font color on revised text
\newcommand{\fn}[1]{\footnote{#1}} %shortcut for footnote tag
\newcommand*\sq{\mathbin{\vcenter{\hbox{\rule{.3ex}{.3ex}}}}} %makes a small square as a separator $\sq$
%\newcommand{\sk}[1]{\sout{#1}} %shortcut for default strikethrough - do not sk through citep
\newcommand\sk{\bgroup\markoverwith{\textcolor{red}{\rule[0.5ex]{1pt}{1pt}}}\ULon} %strikethrough with red line; not in \section{}
%\st{} does strikethrough using soul package but does not like acronyms
\newcommand{\blucell}{\cellcolor{aliceblue}} %use to shade in table cell
\newcommand{\grycekk}{\cellcolor{lightgray}} %use to shade in table cell
\newcommand{\whicell}{\cellcolor{antiquewhite}} %use to shade in table cell
%%%%%


%%%%% colors
%http://latexcolor.com/
%https://en.wikibooks.org/wiki/LaTeX/Colors#:~:text=black%2C%20blue%2C%20brown%2C%20cyan,be%20available%20on%20all%20systems.
\definecolor{aliceblue}{rgb}{0.94, 0.97, 1.0}
\definecolor{antiquewhite}{rgb}{0.98, 0.92, 0.84}
\definecolor{lightmauve}{rgb}{0.86, 0.82, 1.0}
\definecolor{brilliantlavender}{rgb}{0.96, 0.73, 1.0}
\definecolor{brandeisblue}{rgb}{0.0, 0.44, 1.0}
\definecolor{darkmidnightblue}{rgb}{0.0, 0.2, 0.4}
%%%%%


%%%%% acronyms
\newcommand{\acf}{\acrfull} %full acronym
\newcommand{\acl}{\acrlong} %long acronym
\newcommand{\acs}{\acrshort} %short acronym

\newcommand{\acfp}{\acrfullpl} %full acronym plural
\newcommand{\aclp}{\acrlongpl} %long acronym plural
\newcommand{\acsp}{\acrshortpl} %short acronym plural
%%%%%


%%%%% todonotes
\newcommandx{\cmt}[2][1=]{\todo[author=\textbf{STRUCTURE},tickmarkheight=0.15cm,linecolor=red,backgroundcolor=red!25,bordercolor=black,#1]{#2}}
\newcommandx{\con}[2][1=]{\todo[author=\textbf{CONTENT},tickmarkheight=0.15cm,linecolor=brilliantlavender,backgroundcolor=brilliantlavender,bordercolor=black,#1]{#2}}
\newcommandx{\rab}[2][1=]{\todo[noline,author=\textbf{RAB},backgroundcolor=Plum!25,bordercolor=black,#1]{#2}}
\newcommandx{\one}[2][1=]{\todo[author=\textbf{Reviewer 1},tickmarkheight=0.15cm,linecolor=OliveGreen,backgroundcolor=OliveGreen!25,bordercolor=black,#1]{#2}} 
\newcommandx{\two}[2][1=]{\todo[author=\textbf{Reviewer 2},tickmarkheight=0.15cm,linecolor=Plum,backgroundcolor=Plum!25,bordercolor=black,#1]{#2}}
\newcommandx{\thr}[2][1=]{\todo[author=\textbf{Reviewer 3},tickmarkheight=0.15cm,linecolor=Plum,backgroundcolor=Plum!25,bordercolor=black,#1]{#2}}

%\newcommandx{\jon}[2][1=]{\todo[noline,author=\textbf{ATTN: Johnson},backgroundcolor=blue!25,bordercolor=black,#1]{#2}}
%\newcommandx{\han}[2][1=]{\todo[noline,author=\textbf{ATTN: Haney},backgroundcolor=OliveGreen!25,bordercolor=black,#1]{#2}}
%\newcommandx{\rab}[2][1=]{\todo[author=\textbf{RAB},tickmarkheight=0.15cm,linecolor=Plum,backgroundcolor=Plum!25,bordercolor=black,#1]{#2}}
%\newcommandx{\han}[2][1=]{\todo[author=\textbf{ATTN: Haney},tickmarkheight=0.15cm,linecolor=OliveGreen,backgroundcolor=OliveGreen!25,bordercolor=OliveGreen,#1]{#2}}
%\newcommandx{\jon}[2][1=]{\todo[author=\textbf{ATTN: Johnson},tickmarkheight=0.15cm,linecolor=blue,backgroundcolor=blue!25,bordercolor=blue,#1]{#2}}


% highlighting 
\DeclareRobustCommand{\hlc}[1]{{\sethlcolor{LimeGreen}\hl{#1}}}
\makeatletter
    \if@todonotes@disabled
    \newcommand{\hlh}[2]{#1}
    \else
    \newcommand{\hlh}[2]{\han{#2}\hlc{#1}}
    \fi
    \makeatother

\DeclareRobustCommand{\hld}[1]{{\sethlcolor{CornflowerBlue}\hl{#1}}}
\makeatletter
    \if@todonotes@disabled
    \newcommand{\hlj}[2]{#1}
    \else
    \newcommand{\hlj}[2]{\jon{#2}\hld{#1}}
    \fi
    \makeatother

\DeclareRobustCommand{\hlf}[1]{{\sethlcolor{lightmauve}\hl{#1}}}
\makeatletter
    \if@todonotes@disabled
    \newcommand{\hlb}[2]{#1}
    \else
    \newcommand{\hlb}[2]{\rab{#2}\hlf{#1}}
    \fi
    \makeatother
%%%%%


%%%%% table alignments
\newcolumntype{L}[1]{>{\raggedright\let\newline\\\arraybackslash\hspace{0pt}}m{#1}} %uses \raggedright with m,p{} in table column
\newcolumntype{C}[1]{>{\centering\let\newline\\\arraybackslash\hspace{0pt}}m{#1}} %uses \raggedright with m,p{} in table column
\newcolumntype{R}[1]{>{\raggedleft\let\newline\\\arraybackslash\hspace{0pt}}m{#1}} %uses \raggedright with m,p{} in table column
%%%%%


%%%%% table contents
\makeatletter
\renewcommand\tableofcontents{%
    \@starttoc{toc}%
}
\makeatother

\makeatletter
\renewcommand\listoffigures{%
    \@starttoc{lof}%
}
\makeatother

\makeatletter
\renewcommand\listoftables{%
    \@starttoc{lot}%
}
\makeatother

\makeatletter
\newcommand*\ftp{\fontsize{16.5}{17.5}\selectfont}
\makeatother
%%%%%


%%%%% archived section commands - use titlesec
%\makeatletter
%\renewcommand\section{%
%    \@startsection{section}{1}{\z@ }{0.50\baselineskip}{0.25\baselineskip}
%    {\large \normalfont \bfseries}}%
%\makeatletter
%\renewcommand\paragraph{%
%    \@startsection{paragraph}{4}{\z@ }{0.55\baselineskip}{-1em}
%    {\normalfont \normalsize \bfseries}}%
%\makeatletter
%\renewcommand\subparagraph{%
%    \@startsection{subparagraph}{5}{\z@ }{0.40\baselineskip}{-1em}
%    {\normalfont \normalsize \itshape }}%
%\makeatletter
%\renewcommand\subsection{%
%    \@startsection{subsection}{2}{\z@ }{0.45\baselineskip}{0.25\baselineskip}
%    {\large \normalfont \bfseries}}%
%%%%%


%%%%% header and footer
\usepackage{fancyhdr}
\pagestyle{fancy}
\fancyhf{} %move page number to bottom right
%\renewcommand{\headrulewidth}{0pt} %turn off line in header
\lhead{\scriptsize NE585}
\chead{\scriptsize \textit{Project 5 -- Monte Carlo methods and MCNP}}
\rhead{\scriptsize \today}
\rfoot{\thepage}
%%%%%


%%%%% citations
\begin{filecontents}{references.bib}
    @misc{
        nye15a,
        author = {Nye, Jr., Joseph S.},
        title = {{The World Needs an Arms-control Treaty for Cybersecurity}},
        year = {2015},
        note = {{The Washington Post}}
    }
    @article{
        mil20a,
        author = {Miller, James N.},
        title = {No to no first use — for now},
        journal = {Bulletin of the Atomic Scientists},
        volume = {76},
        pages = {8},
        year  = {2020}
    }
    @incollection{
        mil17a,
        author = {Miller, Steven E.},
        title = {{Cyber Threats, Nuclear Analogies? Divergent Trajectories in Adapting to New Dual-Use Technologies}},
        booktitle = {Understanding Cyber Conflict: 14 Analogies},
        publisher = {Georgetown University Press},
        year = {2017},
        editor = {Perkovich, George and Levite, Ariel E.},
        pages = {161}
    }
    @conference{
        ben20a,
        author = {Benjamin, Jacob and Haney, Michael},
        title = {{Nonproliferation of Cyber Weapons}},
        year = { 2020},
        organization = {Symposium on Cyber Warfare, Cyber Defense, \& Cyber Security },
        address = {Las Vegas, Nevada}
    }
\end{filecontents}
%%%%%


%%%% acronyms
% alphabetical ordering is automated
\newacronym{nrs}{NRHES}{Nuclear Renewable Hybrid Energy System}
\newacronym{ahp}{AHP}{Analytical Hierarchy Process}
\newacronym{inl}{INL}{Idaho National Laboratory}
\newacronym{orl}{ORNL}{Oak Ridge National Laboratory}
\newacronym{anl}{ANL}{Argonne National Laboratory}
\newacronym{npp}{NPP}{Nuclear Power Plant}
\newacronym{smr}{SMR}{Small Modular Reactor}
\newacronym{ump}{UAMPS}{Utah Associated Municipal Power Systems}
\newacronym{nus}{NuScale}{NuScale Power, LLC}
\newacronym{nrc}{NRC}{United States Nuclear Regulatory Commission}
\newacronym{epri}{EPRI}{Electric Power Research Institute}
\newacronym{nerc}{NERC}{North American Electric Reliability Corporation}
\newacronym{ci}{CI}{Consistency Index}
\newacronym{cr}{CR}{Consistency Ratio}
\newacronym{htse}{HTSE}{High Temperature Steam Electrolysis}
\newacronym{lwr}{LWR}{Light Water Reactor}
\newacronym{eia}{EIA}{U.S. Energy Information Administration}
\newacronym{oer}{OER}{Online Educational Resource}
\newacronym{lms}{LMS}{Learning Management System}
\newacronym{cps}{CPS}{Cyber-Physical Systems}
\newacronym{nsf}{NSF}{National Science Foundation}
\newacronym{wsc}{WSC}{Western Services Corporation}
\newacronym{cae}{CAES}{Center for Advanced Energy Studies}
\newacronym{hsl}{HSSL}{Human System Simulation Laboratory}
\newacronym{pwr}{PWR}{Pressurized Water Reactor}
\newacronym{bwr}{BWR}{Boiling Water Reactor}
\newacronym{roi}{ROI}{Return on Investment}
\newacronym{ic}{I\&C}{Instrumentation \& Controls}
\newacronym{mwe}{MWe}{Megawatts-electric}
\newacronym{ics}{ICS}{Industrial Control Systems}
\newacronym{sca}{SCADA}{Supervisory Control and Data Acquisition}
\newacronym{ip}{IP}{Internet Protocol}
\newacronym{udp}{UDP}{User Datagram Protocol}
\newacronym{tva}{TVA}{Tennessee Valley Authority}
\newacronym{plc}{PLC}{Programmable Logic Controller}
\newacronym{vfd}{VFD}{Variable Frequency Drive}
\newacronym{khp}{KHNP}{Korean Hydro \& Nuclear Power Co., Ltd}
\newacronym{onl}{ORNL}{Oak Ridge National Laboratory}
\newacronym{jcp}{JCPOA}{Joint Comprehensive Plan of Action}
\newacronym{mim}{MITM}{Man in the Middle}
\newacronym{dos}{DDoS}{Distributed Denial of Service}
\newacronym{tcp}{TCP/IP}{Transmission Control Protocol/Internet Protocol}
\newacronym{dnp}{DNP3}{Distributed Network Protocol 3}
\newacronym{pra}{PRA}{Probabilistic Risk Assessment}
\newacronym{cs}{CS}{Critical System}
\newacronym{loc}{LOCA}{Loss of Coolant Accident}
\newacronym{hmi}{HMI}{Human Machine Interface}
\newacronym{pha}{PHA}{Preliminary Hazards Analysis}
\newacronym{bol}{BOL}{Beginning-of-Life}
\newacronym{eol}{EOL}{End-of-Life}
\newacronym{mol}{MOL}{Middle-of-Life}
\newacronym{imu}{IMUNES}{Integrated Multiprotocol Network Emulator/Simulator}
\newacronym{ccc}{CCC}{Computing Community Consortium}
\newacronym{neu}{NEUP}{Nuclear Energy University Program}
\newacronym{doe}{DOE}{United States Department of Energy}
\newacronym{nei}{NEI}{Nuclear Energy Institute}
\newacronym{nit}{NITRD}{Networking Information Technology Research \& Development Program}
\newacronym{rcs}{RCS}{Reactor Cooling System}
\newacronym{con}{IC}{Initial Condition}
\newacronym{csi}{CSIS}{Center for Strategic \& International Studies}
\newacronym{pcap}{PCAP}{packet capture file}
\newacronym{dc}{DC}{Direct-Current}
\newacronym{ac}{AC}{Alternating-Current}
\newacronym{iff}{UIIF}{Idaho Falls Center for Higher Education}
\newacronym{snl}{SNL}{Sandia National Laboratory}
\newacronym{cie}{CIE}{Cyber-Informed Engineering}
\newacronym{cds}{CRDS}{Control Rod Drive System}
\newacronym{cdm}{CRDM}{Control Rod Drive Mechanism}
\newacronym{fma}{FMEA}{Failure Modes \& Effects Analysis}
\newacronym{rpn}{RPN}{Risk Priority Number}
\newacronym{scr}{SCR}{silicon controller rectifier}
\newacronym{hvc}{HVAC}{Heating, Ventilation \& Air Conditioning}
\newacronym{ttb}{TTB}{Time-to-Boil}
\newacronym{sis}{SIS}{Safety Instrumented System}
\newacronym{usb}{USB}{Universal Serial Bus}
\newacronym{ack}{ACK}{Acknowledge}
\newacronym{seq}{SEQ}{Sequence}
\newacronym{npr}{NPR}{National Public Radio}
\newacronym{cyd}{CYDEF}{Cyber Defence QCD Corporation}
\newacronym{ara}{ARA}{Amsterdam-Rotterdam-Antwerp}
%\newacronym{}{}{}
%%%%%


\begin{document}

\begin{titlepage}
    \title{
        NE585 -- Nuclear fuel cycle analysis\\
        Project 5 -- Monte Carlo methods and MCNP\\
    }
    \author{
        Name
        \\ \\ \\
        University of Idaho $\sq$ Idaho Falls Center for Higher Education
        \\ \\
        Nuclear Engineering and Industrial Management Department
        \\ \\ \\
        email 
    }
\clearpage %not have page number on title page
\maketitle
\vspace*{\fill}
\begin{flushright}{
        Total -- 420 
}
\end{flushright}
\thispagestyle{empty} %start with page number 1 on second page
\end{titlepage}

\printnoidxglossary

\newpage

\section*{Preface}
For problems 1 -- 3 use Monte Carlo techniques to obtain the solutions.

\newpage

\section{Evaluating an integral I}
\paragraph*{(30)}
Solve for G using Monte Carlo techniques. Solve the integral analytically and graph g(x). Also plot G v N for $N = 10,\;10^2.\;10^3,\;10^4,\;10^5,\;10^6$.
\begin{equation}
    G = \int_0^1 g(x)dx 
\end{equation}

\begin{equation}
    g(x)=1-e^{-x}
\end{equation}





\newpage

\section{Evaluating an integral II}
\paragraph*{(30)}
Do the same for the following function. This g(x) does not have an analytical solution. However, use a numerical solver or other technique to compare the Monte Carlo result. Graph g(x) and plot G v N.
\begin{equation}
    \int_0^{\frac{\pi}{2}}sin(x^2)dx
\end{equation}





\newpage

\section{Approximation}
\paragraph*{(30)}
Approximate $\sqrt{2}$ in a similar manner to the way we approximated $\pi$. Plot the convergence. 





\newpage

\section{Hot cell modeling}
\paragraph*{(50)}
Conduct a short modeling study of the metal fuel alloy for the hot cell facility using - 
\begin{itemize}[leftmargin=*,topsep=0pt]
    \item\href{https://github.com/TheDoctorRAB/mcnpx.decks/blob/master/neutron.flux/input/4_ff.alloy.inp}{4\_ff.alloy.inp (flux)}
    \item\href{https://github.com/TheDoctorRAB/mcnpx.decks/blob/master/neutron.flux/input/4d_ff.alloy.inp}{4d\_ff.alloy.inp (dose)}
\end{itemize}

\vspace{\baselineskip}

\noindent See also the related \href{https://www.sciencedirect.com/science/article/abs/pii/S0029549316301066}{paper on hot cell shielding} for more information. 

\vspace{\baselineskip}

\noindent Apply the following procedure -- 
\begin{enumerate}[leftmargin=*,topsep=0pt,label=(\alph*)]
    \item Look at the original geometry in the plotter/VisEd.
    \item Modify the facility to only include the SE and SW cells.
    \item Use MCNP to compute the volume averaged (F4) flux tallies for the SE cell and SW cell.
    \item Use a neutron emission rate of of $1.1 \times 10^7 \; n/s/g$ for 24 grams of material. 
    \item Increase NPS from the original files to reduce standard error.
    \item Start with a wall thickness of 15 cm using the material already included in the deck. It is a form of borated concrete that is common to these kinds of facilities. Increase the wall thickness until the dose rate falls below $1 \; \mu Sv/h$ and the relative flux falls below 0.01. \textit{There seems to be some issues with dose card IC = 10. Appendix F should have a fix. Try to see if it will work.}
    \item Plot dose rate v wall thickness and the relative flux v wall thickness.
    \item Justify that the results are scientifically sound.
\end{enumerate}

\vspace{\baselineskip}

\noindent\textit{Is this wall thickness reasonable? As in, could a facility be practically built like this using current engineering design techniques?}

\vspace{\baselineskip}

\noindent\textit{Include the MCNP file at the end in an appendix.}





\newpage

\section*{Criticality modeling}
For the criticality models, to get full credit -- 
\begin{enumerate}[leftmargin=*,topsep=0pt,label=(\roman*)]
    \item Include a screenshot of the model from the VisEd/plotter.
    \item Use finite geometries.
    \item Design geometries that will minimize leakage. Show (as part of making the mcnp file and results; not calculating by hand) that leakage has been minimized.
    \item For criticality, get to 3 9s or 0s (.999x, 1.000x) for the mean for full credit. Bonus for 4 9s/0s.
    \item Report output in a table -- k, standard deviation, 95\% confidence. 
    \item Justify the results are scientifically sound.
    \item Include the input deck in the appendix.
\end{enumerate}

\vspace{\baselineskip}

\noindent\textbf{PROTIP -- }k can vary weird when your trying to get the critical radius to 4 or 5 decimal places. Study the KCODE parameters. You could also add more particles on KSRC, but be careful where you place them.

\newpage

\section{Critical mass I}
\paragraph*{(20)}
What is the critical mass of a bare sphere of plutonium containing (1) $95.5\% \; ^{239}Pu$ and (2) $80\% \; ^{239}Pu$, where the rest is $^{238}Pu$? \textit{Comment on the results.}





\newpage

\section{Critical mass with reflector}
\paragraph*{(20)}
What is the critical mass for the above, but with a thin nickel shell of 0.10 cm?





\newpage

\section{Critical mass II}
\paragraph*{(20)}
What is the critical mass of pure $^{239}Pu$ of a bare cylinder?





\newpage

\section*{Reflector modeling}
For the next three problems, select 3 -- 5 typical reflector materials for each fissionable source. Make a table for each source with results from the reflectors.

\newpage

\section{Critical mass I}
\paragraph*{(30)}
Taking the bare sphere $^{239}Pu$ model, what is the `optimal' reflector that minimizes the \textit{critical mass}? Pick at least one reflector material that is `exotic'; e.g., maybe for a reactor powering a Mars Rover style robot mining on an asteroid or a moon.  





\newpage

\section{Critical mass II}
\paragraph*{(30)}
Do the same for $^{235}U$.





\newpage

\section{Critical mass III}
\paragraph*{(30)}
Do the same for $^{233}U$.





\newpage

\section{Reflector analysis}
\paragraph*{(30)}
Put all the results from the reflector problems together in a table. Which reflector material is minimal and why, neutronically speaking? Check the output files for different types of neutron losses as part of the analysis. Comment on cost in the analysis of optimal material, generally estimate; no need to research specific costs.





\newpage

\section{Geometry challenge}
\paragraph*{(50)}
Three unreflected aluminum cylinders contain $U(93.2)O_2F_2$ water solutions. The inside cylinder diameter and critical height measured 20.3 cm and 41.4 cm. The aluminum container had a density of $2.71 \; g/cm^3$ and was 0.15 cm thick. The three cylinders were set in an equilateral configuration with a surface separation of 0.38 cm. The solution concentration parameters were $0.90 \; g(^{235}U)/cm^3$ with $H:^{235}U = 309$.

\begin{enumerate}[leftmargin=*,topsep=0pt,label=(\alph*)]
    \item It was estimated that the solution density was approximately $1.131 \; g/cm^3$ and consisted of\\ $0.0021345\;^{235}U, \; 0.00015382\;^{238}U, \; 0.33383\;O, \; 0.65930\;H, \; 0.0045756\;F \; atoms/b-cm$.
    \item MCNP gives $k = 0.9991\pm0.0011$. \textit{Get within 15\% for full credit.}
\end{enumerate}    

\vspace{\baselineskip}

\noindent Reproduce the model to get the result. See the \href{https://uidaho.pressbooks.pub/nuclearengineering/chapter/mcnp/}{MCNP benchmark document} for guidance.





\newpage

\section{Critical mass IV}
\paragraph*{(30)}
Find the critical mass for a bare cylinder of 10.9\% enriched U with a density of 18.63 g/cc. Optimize the radius and height for neutronics, again based on neutron losses. 





\newpage

\section{Critical mass V}
\paragraph*{(20)}
Find the minimum critical mass for an infinite graphite reflected 93.5\% enriched U sphere. Use 18.8 g/cc for the U density and just use carbon for the graphite.





\newpage

\section{Critical mass VI}
\paragraph*{(20)}
Find the critical mass of 97.67\% enriched U cube in an infinite water reflector. Use a density of 18.794 g/cc.





\newpage

\section*{Tables}

{%
\let\oldnumberline\numberline%
\renewcommand{\numberline}{\tablename~\oldnumberline}%
\listoftables%
}

\newpage
%Put tables here

%\setlength{\arrayrulewidth}{0.3mm}
%\begin{longtable}{|l|c|c|}
%    \caption{Gas processing constraints}
%    \label{tab-gas-processing}
%    \\
%    \hline
%    \textbf{Constraint}
%    &\textbf{Regular grade}
%    &\textbf{Premium grade}
%    \\
%    \hline
%    Raw gas
%    & 7 m\textsuperscript{3}/ton
%    & 11 m\textsuperscript{3}/ton
%    \\
%    \hline
%    Production time
%    & 10 h/ton
%    & 8 h/ton
%    \\
%    \hline
%    Storage
%    & 9 tons
%    & 6 tons
%    \\
%    \hline
%    Profit
%    & \$150/ton
%    & \$175/ton
%    \\
%    \hline
%\end{longtable}

\newpage

\section*{Figures}

{%
\let\oldnumberline\numberline%
\renewcommand{\numberline}{\figurename~\oldnumberline}%
\listoffigures%
}

\newpage
%Put figures here

%\begin{figure}[h!]
%    \centering
%    \includegraphics[width=0.X\textwidth]{filename}
%    \caption{Descriptive caption}
%    \label{fig-label-name}
%\end{figure}

\newpage

%appendix - uncomment below to add - do not uncomment this line

\begin{appendices}

    \renewcommand{\thesection}{\Roman{section}}
    \titleformat{\section}{\normalfont\large\bfseries\filcenter}{Appendix \thesection:}{1ex}{}[]

    \section{Hot cell MCNP input decks} \label{app-hot-cell}

\newpage

    \section{Criticality MCNP input decks} \label{app-crit}

\end{appendices}

\newpage 

\bibliographystyle{nsf}
\setlength{\bibhang}{0pt}
\bibliography{references}

\end{document}
