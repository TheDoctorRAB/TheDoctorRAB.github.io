\documentclass[11pt,a4paper]{article}
\usepackage[landscape,lmargin=1in,rmargin=1in,tmargin=1in,bmargin=1in]{geometry}
\usepackage[pagewise]{lineno} %line numbering
\usepackage{setspace}
\usepackage{ulem} %strikethrough - do not \sout{\cite{}}
\usepackage{xcolor} %change font color
\usepackage{graphicx}
\usepackage{filecontents}
\usepackage{tablefootnote}
\usepackage{footnotehyper}
\usepackage{subfig}
\usepackage[yyyymmdd]{datetime} %date format
\renewcommand{\dateseparator}{.}
\graphicspath{{../img/}} %path to graphics
\setcounter{secnumdepth}{5} %set subsection to nth level
\usepackage{caption}
\captionsetup[table]{skip=11pt} %sets a space after table caption
\usepackage{times}
\usepackage{tabto} %general tabbed spacing
\usepackage{longtable} %need to put label at top under caption then \\ - use spacing
\usepackage[stable,hang,flushmargin]{footmisc} %footnotes in section titles and no indent
\usepackage[round]{natbib} %parenthesis instead of brackets for inline citations
\usepackage{enumitem}
\usepackage{boldline}
\usepackage{makecell}
\usepackage{booktabs}
\usepackage{amssymb}
\usepackage{amsmath}
\usepackage{physics}
\usepackage{tabularx}
\usepackage{multirow}
\usepackage{lscape}
\usepackage{array}
\usepackage{caption}
\usepackage[labelfont=bf]{caption}
\usepackage{chngcntr}
\usepackage{hyperref}

\newcommand{\edit}[1]{\textcolor{blue}{#1}} %shortcut for changing font color on revised text
\newcommand{\fn}[1]{\footnote{#1}} %shortcut for footnote tag
\newcommand*\sq{\mathbin{\vcenter{\hbox{\rule{.3ex}{.3ex}}}}} %makes a small square as a separator $\sq$
\renewcommand\labelenumi{(\theenumi)} %changes 1. to (1) in enumerated list

\usepackage{fancyhdr}
\pagestyle{fancy}
\fancyhf{} %move page number to bottom right
\renewcommand{\headrulewidth}{0.5pt} %turn off line in header
\lhead{\scriptsize }
\chead{\scriptsize Xenon neutron poison production}
\rhead{\scriptsize }
\rfoot{\thepage}

\begin{document}

\begin{titlepage}
    \title{
        NE504 - Nuclear fuel cycle analysis\\
        Xenon neutron poison production\\
    }
    \author{
        R. A. Borrelli
        \\ \\ \\
        University of Idaho $\sq$ Idaho Falls Center for Higher Education
        \\ \\
        Nuclear Engineering and Industrial Management Department
        \\ \\ \\
        r.angelo.borrelli@gmail.com
    }
\clearpage %not have page number on title page
\maketitle
\thispagestyle{empty} %start with page number 1 on second page
\end{titlepage}

\section{Mathematical models}
\subsection{Iodine}
Rate of change of iodine - 
\begin{equation} \label{eq-i-dq}
    \begin{gathered}
        \frac{dI}{dt}=\gamma_I \Sigma_F \phi_T-\lambda_I I \\
        I(0)=0
    \end{gathered}
\end{equation}

\subsection{Xenon}
Rate of change of xenon - 
\begin{equation} \label{eq-xe-dq}
    \begin{gathered}
        \frac{dX}{dt}=\lambda_I I + \gamma_X \Sigma_F \phi_T - \lambda_X X - \sigma_A \phi_T X \\
        X(0)=0
    \end{gathered}
\end{equation}

\newpage

\section{Solutions}
\subsection{Iodine}
Apply laplace transform to eq. \ref{eq-i-dq} - 
\begin{equation} \label{eq-i-laplace-transform}
    \begin{split}
        & \frac{dI}{dt}=\gamma_I \Sigma_F \phi_T-\lambda_I I \\
        & s\tilde{I}=\frac{1}{s}\gamma_I\Sigma_F\phi_T-\lambda_I\tilde{I}
    \end{split}
\end{equation}
\vspace{\baselineskip}

Rearrange terms to obtain laplace solution - 
\begin{equation} \label{eq-i-laplace-solution}
    \tilde{I}=\frac{1}{s(s+\lambda_I)}\gamma_I\Sigma_F\phi_T
\end{equation}
\vspace{\baselineskip}

Invert eq. \ref{eq-i-laplace-solution} to obtain real time domain solution - 
\begin{equation} \label{eq-i-solution}
    I(t)=\frac{\gamma_I\Sigma_F\phi_T}{\lambda_I}(1-e^{-\lambda_I t})
\end{equation}

\subsection{Xenon}
Substitute the iodine solution in eq. \ref{eq-i-solution} into eq. \ref{eq-xe-dq} - 
\begin{equation} 
    \begin{split}
        & \frac{dX}{dt}=\lambda_I I + \gamma_X \Sigma_F \phi_T - \lambda_X X - \sigma_A \phi_T X \\
        & \frac{dX}{dt}=\lambda_I \frac{\gamma_I\Sigma_F\phi_T}{\lambda_I}(1-e^{-\lambda_I t}) + \gamma_X \Sigma_F \phi_T - \lambda_X X - \sigma_A \phi_T X
    \end{split}
\end{equation}
\vspace{\baselineskip}

Rearrange the terms and simplify - 
\begin{equation} 
    \begin{split}
        & \frac{dX}{dt}=\gamma_I\Sigma_F\phi_T(1-e^{-\lambda_I t}) + \gamma_X \Sigma_F \phi_T - \lambda_X X - \sigma_A \phi_T X \\
        & \frac{dX}{dt}=\gamma_I\Sigma_F\phi_T - \gamma_I\Sigma_F\phi_T e^{-\lambda_I t} + \gamma_X \Sigma_F \phi_T - \lambda_X X - \sigma_A \phi_T X 
    \end{split}
\end{equation}
\vspace{\baselineskip}

\begin{equation} \label{eq-xe-dq-full}
    \frac{dX}{dt}=\gamma_I\Sigma_F\phi_T - \gamma_I\Sigma_F\phi_T e^{-\lambda_I t} + \gamma_X \Sigma_F \phi_T - (\lambda_X + \sigma_A \phi_T) X
\end{equation}
\vspace{\baselineskip}

Apply laplace transform to eq. \ref{eq-xe-dq-full} -
\begin{equation} \label{eq-xe-laplace-transform}
    s\tilde{X}=\frac{1}{s}\gamma_I\Sigma_F\phi_T - \frac{1}{s+\lambda_I}\gamma_I\Sigma_F\phi_T + \frac{1}{s}\gamma_X\Sigma_F\phi_T - (\lambda_X+\sigma_A\phi_T)\tilde{X}
\end{equation}
\vspace{\baselineskip}

Rearrange the terms in eq. \ref{eq-xe-laplace-transform} to obtain the laplace solution - 
\begin{equation}
    (s+[\lambda_X+\sigma_A\phi_T])\tilde{X}=\frac{1}{s}\gamma_I\Sigma_F\phi_T - \frac{1}{s+\lambda_I}\gamma_I\Sigma_F\phi_T + \frac{1}{s}\gamma_X\Sigma_F\phi_T
\end{equation}
\vspace{\baselineskip}

Then - 
\begin{equation} \label{eq-xe-laplace-solution} 
        \tilde{X}=
        +\frac{1}{s(s+[\lambda_X+\sigma_A\phi_T])}\gamma_I\Sigma_F\phi_T
        -\frac{1}{(s+\lambda_I)(s+[\lambda_X+\sigma_A\phi_T])}\gamma_I\Sigma_F\phi_T 
        +\frac{1}{s(s+[\lambda_X+\sigma_A\phi_T])}\gamma_X\Sigma_F\phi_T
\end{equation}
\vspace{\baselineskip}

Invert eq. \ref{eq-xe-laplace-solution} to obtain real time domain solution - 
\begin{equation} \label{eq-xe-solution} 
        X(t)=
        \frac{\gamma_I\Sigma_F\phi_T}{\lambda_X+\sigma_A\phi_T}(1-e^{-(\lambda_X+\sigma_A\phi_T)t}) 
        -\frac{\gamma_I\Sigma_F\phi_T}{\lambda_X+\sigma_A\phi_T-\lambda_I}(e^{-\lambda_I t}-e^{-(\lambda_X+\sigma_A\phi_T)t}) 
        +\frac{\gamma_X\Sigma_F\phi_T}{\lambda_X+\sigma_A\phi_T}(1-e^{-(\lambda_X+\sigma_A\phi_T)t}) 
\end{equation}

\newpage

\section{Derivatives}
\subsection{Iodine}
\begin{equation} \label{eq-i-solution}
    I(t)=\frac{\gamma_I\Sigma_F\phi_T}{\lambda_I}(1-e^{-\lambda_I t})
\end{equation}

Expand - 
\begin{equation} \label{eq-i-solution}
    I(t)=\frac{\gamma_I\Sigma_F\phi_T}{\lambda_I}-\frac{\gamma_I\Sigma_F\phi_T}{\lambda_I}e^{-\lambda_I t}
\end{equation}

Compute derivative - 
\begin{equation}
    \begin{gathered}
        \frac{dI}{dt}=
        \frac{d}{dt}[\frac{\gamma_I\Sigma_F\phi_T}{\lambda_I}]
        -\frac{d}{dt}[\frac{\gamma_I\Sigma_F\phi_T}{\lambda_I}e^{-\lambda_I t}]\\
        \frac{dI}{dt}=
        0 - (-\lambda_I)(\frac{\gamma_I\Sigma_F\phi_T}{\lambda_I}e^{-\lambda_I t})
    \end{gathered}
\end{equation}

Therefore - 
\begin{equation}
        \frac{dI}{dt}=
        \gamma_I\Sigma_F\phi_T e^{-\lambda_I t}
\end{equation}

\subsection{Xenon}
\begin{equation} \label{eq-xe-solution}
        X(t)=
        \frac{\gamma_I\Sigma_F\phi_T}{\lambda_X+\sigma_A\phi_T}(1-e^{-(\lambda_X+\sigma_A\phi_T)t}) 
        -\frac{\gamma_I\Sigma_F\phi_T}{\lambda_X+\sigma_A\phi_T-\lambda_I}(e^{-\lambda_I t}-e^{-(\lambda_X+\sigma_A\phi_T)t}) 
        +\frac{\gamma_X\Sigma_F\phi_T}{\lambda_X+\sigma_A\phi_T}(1-e^{-(\lambda_X+\sigma_A\phi_T)t}) 
\end{equation}

Rearrange to make computing the derivative easier - 
\begin{equation} 
    \begin{aligned}
        X(t)=\\
        &(\frac{\gamma_I\Sigma_F\phi_T}{\lambda_X+\sigma_A\phi_T}-\frac{\gamma_I\Sigma_F\phi_T}{\lambda_X+\sigma_A\phi_T}e^{-(\lambda_X+\sigma_A\phi_T)t})\\
        &-(\frac{\gamma_I\Sigma_F\phi_T}{\lambda_X+\sigma_A\phi_T-\lambda_I}e^{-\lambda_I t}
        -\frac{\gamma_I\Sigma_F\phi_T}{\lambda_X+\sigma_A\phi_T-\lambda_I}e^{-(\lambda_X+\sigma_A\phi_T)t})\\
        &+(\frac{\gamma_X\Sigma_F\phi_T}{\lambda_X+\sigma_A\phi_T}-\frac{\gamma_X\Sigma_F\phi_T}{\lambda_X+\sigma_A\phi_T}e^{-(\lambda_X+\sigma_A\phi_T)t}) 
    \end{aligned}
\end{equation}

\begin{equation} 
    \begin{aligned}
        X(t)=\\
        &(\frac{\gamma_I\Sigma_F\phi_T}{\lambda_X+\sigma_A\phi_T}-\frac{\gamma_I\Sigma_F\phi_T}{\lambda_X+\sigma_A\phi_T}e^{-(\lambda_X+\sigma_A\phi_T)t})\\
        &-(\frac{\gamma_I\Sigma_F\phi_T}{\lambda_X+\sigma_A\phi_T-\lambda_I}e^{-\lambda_I t}
        -\frac{\gamma_I\Sigma_F\phi_T}{\lambda_X+\sigma_A\phi_T-\lambda_I}e^{-(\lambda_X+\sigma_A\phi_T)t})\\
        &+(\frac{\gamma_X\Sigma_F\phi_T}{\lambda_X+\sigma_A\phi_T}-\frac{\gamma_X\Sigma_F\phi_T}{\lambda_X+\sigma_A\phi_T}e^{-(\lambda_X+\sigma_A\phi_T)t}) 
    \end{aligned}
\end{equation}

\begin{equation} 
    \begin{aligned}
        X(t)=\\
        &\frac{\gamma_I\Sigma_F\phi_T}{\lambda_X+\sigma_A\phi_T}-\frac{\gamma_I\Sigma_F\phi_T}{\lambda_X+\sigma_A\phi_T}e^{-(\lambda_X+\sigma_A\phi_T)t}\\
        &-\frac{\gamma_I\Sigma_F\phi_T}{\lambda_X+\sigma_A\phi_T-\lambda_I}e^{-\lambda_I t}
        +\frac{\gamma_I\Sigma_F\phi_T}{\lambda_X+\sigma_A\phi_T-\lambda_I}e^{-(\lambda_X+\sigma_A\phi_T)t}\\
        &+\frac{\gamma_X\Sigma_F\phi_T}{\lambda_X+\sigma_A\phi_T}
        -\frac{\gamma_X\Sigma_F\phi_T}{\lambda_X+\sigma_A\phi_T}e^{-(\lambda_X+\sigma_A\phi_T)t} 
    \end{aligned}
\end{equation}

Group the terms by exponent - 
\begin{equation} 
    \begin{aligned}
        X(t)=\\
        &\frac{\gamma_I\Sigma_F\phi_T}{\lambda_X+\sigma_A\phi_T}+\frac{\gamma_X\Sigma_F\phi_T}{\lambda_X+\sigma_A\phi_T}\\
        &-\frac{\gamma_I\Sigma_F\phi_T}{\lambda_X+\sigma_A\phi_T-\lambda_I}e^{-\lambda_I t}\\
        &+\frac{\gamma_I\Sigma_F\phi_T}{\lambda_X+\sigma_A\phi_T-\lambda_I}e^{-(\lambda_X+\sigma_A\phi_T)t}
        -\frac{\gamma_I\Sigma_F\phi_T}{\lambda_X+\sigma_A\phi_T}e^{-(\lambda_X+\sigma_A\phi_T)t}
        -\frac{\gamma_X\Sigma_F\phi_T}{\lambda_X+\sigma_A\phi_T}e^{-(\lambda_X+\sigma_A\phi_T)t} 
    \end{aligned}
\end{equation}

\begin{equation} 
        X(t)=
        \frac{(\gamma_I+\gamma_X)\Sigma_F\phi_T}{\lambda_X+\sigma_A\phi_T}
        -\frac{\gamma_I\Sigma_F\phi_T}{\lambda_X+\sigma_A\phi_T-\lambda_I}e^{-\lambda_I t}
        +\frac{\gamma_I\Sigma_F\phi_T}{\lambda_X+\sigma_A\phi_T-\lambda_I}e^{-(\lambda_X+\sigma_A\phi_T)t}
        -\frac{(\gamma_I+\gamma_X)\Sigma_F\phi_T}{\lambda_X+\sigma_A\phi_T}e^{-(\lambda_X+\sigma_A\phi_T)t}
\end{equation}

\begin{equation} 
        X(t)=
        \frac{(\gamma_I+\gamma_X)\Sigma_F\phi_T}{\lambda_X+\sigma_A\phi_T}
        -\frac{\gamma_I\Sigma_F\phi_T}{\lambda_X+\sigma_A\phi_T-\lambda_I}e^{-\lambda_I t}
        +(\frac{\gamma_I\Sigma_F\phi_T}{\lambda_X+\sigma_A\phi_T-\lambda_I}
        -\frac{(\gamma_I+\gamma_X)\Sigma_F\phi_T}{\lambda_X+\sigma_A\phi_T})e^{-(\lambda_X+\sigma_A\phi_T)t}
\end{equation}

Now compute the derivative - 
\begin{equation} 
    \frac{dX}{dt}=
        \frac{d}{dt}[\frac{(\gamma_I+\gamma_X)\Sigma_F\phi_T}{\lambda_X+\sigma_A\phi_T}]
        -\frac{d}{dt}[\frac{\gamma_I\Sigma_F\phi_T}{\lambda_X+\sigma_A\phi_T-\lambda_I}e^{-\lambda_I t}]
        +\frac{d}{dt}[(\frac{\gamma_I\Sigma_F\phi_T}{\lambda_X+\sigma_A\phi_T-\lambda_I}
        -\frac{(\gamma_I+\gamma_X)\Sigma_F\phi_T}{\lambda_X+\sigma_A\phi_T})e^{-(\lambda_X+\sigma_A\phi_T)t}]
\end{equation}

\begin{equation} 
    \frac{dX}{dt}=
        0
        +(\lambda_I)\frac{\gamma_I\Sigma_F\phi_T}{\lambda_X+\sigma_A\phi_T-\lambda_I}e^{-\lambda_I t}
        -(\lambda_X+\sigma_A\phi_T)(\frac{\gamma_I\Sigma_F\phi_T}{\lambda_X+\sigma_A\phi_T-\lambda_I}
        -\frac{(\gamma_I+\gamma_X)\Sigma_F\phi_T}{\lambda_X+\sigma_A\phi_T})e^{-(\lambda_X+\sigma_A\phi_T)t}
\end{equation}

\begin{equation} 
    \frac{dX}{dt}=
        +(\lambda_I)\frac{\gamma_I\Sigma_F\phi_T}{\lambda_X+\sigma_A\phi_T-\lambda_I}e^{-\lambda_I t}
        -(\lambda_X+\sigma_A\phi_T)(\frac{\gamma_I\Sigma_F\phi_T}{\lambda_X+\sigma_A\phi_T-\lambda_I}
        -\frac{(\gamma_I+\gamma_X)\Sigma_F\phi_T}{\lambda_X+\sigma_A\phi_T})e^{-(\lambda_X+\sigma_A\phi_T)t}
\end{equation}

\begin{equation} 
    \frac{dX}{dt}=
        \frac{\gamma_I\lambda_I\Sigma_F\phi_T}{\lambda_X+\sigma_A\phi_T-\lambda_I}e^{-\lambda_I t}
        -(\lambda_X+\sigma_A\phi_T)(\frac{\gamma_I\Sigma_F\phi_T}{\lambda_X+\sigma_A\phi_T-\lambda_I}
        -\frac{(\gamma_I+\gamma_X)\Sigma_F\phi_T}{\lambda_X+\sigma_A\phi_T})e^{-(\lambda_X+\sigma_A\phi_T)t}
\end{equation}

\begin{equation} 
    \frac{dX}{dt}=
        \frac{\gamma_I\lambda_I\Sigma_F\phi_T}{\lambda_X+\sigma_A\phi_T-\lambda_I}e^{-\lambda_I t}
        -(\frac{(\lambda_X+\sigma_A\phi_T)\gamma_I\Sigma_F\phi_T}{\lambda_X+\sigma_A\phi_T-\lambda_I}
        -\frac{(\lambda_X+\sigma_A\phi_T)(\gamma_I+\gamma_X)\Sigma_F\phi_T}{\lambda_X+\sigma_A\phi_T})e^{-(\lambda_X+\sigma_A\phi_T)t}
\end{equation}

\begin{equation} 
    \frac{dX}{dt}=
        \frac{\gamma_I\lambda_I\Sigma_F\phi_T}{\lambda_X+\sigma_A\phi_T-\lambda_I}e^{-\lambda_I t}
        -(\frac{(\lambda_X+\sigma_A\phi_T)\gamma_I\Sigma_F\phi_T}{\lambda_X+\sigma_A\phi_T-\lambda_I}
        -(\gamma_I+\gamma_X)\Sigma_F\phi_T)e^{-(\lambda_X+\sigma_A\phi_T)t}
\end{equation}

\begin{equation} 
    \frac{dX}{dt}=
        \frac{\gamma_I\lambda_I\Sigma_F\phi_T}{\lambda_X+\sigma_A\phi_T-\lambda_I}e^{-\lambda_I t}
        -\Sigma_F\phi_T(\frac{(\lambda_X+\sigma_A\phi_T)\gamma_I}{\lambda_X+\sigma_A\phi_T-\lambda_I}
        -(\gamma_I+\gamma_X))e^{-(\lambda_X+\sigma_A\phi_T)t}
\end{equation}

\begin{equation} 
    \frac{dX}{dt}=
        \Sigma_F\phi_T\frac{\gamma_I\lambda_I}{\lambda_X+\sigma_A\phi_T-\lambda_I}e^{-\lambda_I t}
        -\Sigma_F\phi_T(\frac{(\lambda_X+\sigma_A\phi_T)\gamma_I}{\lambda_X+\sigma_A\phi_T-\lambda_I}
        -(\gamma_I+\gamma_X))e^{-(\lambda_X+\sigma_A\phi_T)t}
\end{equation}

\newpage

\section{Equilibrium time}
The equilibrium time for xenon should be when the concentration achieves a steady state; i.e., the rate of change in the concentration is zero.

\begin{equation}
        \frac{dI}{dt}=
        \gamma_I\Sigma_F\phi_T e^{-\lambda_I t}
\end{equation}

\begin{equation} 
    \frac{dX}{dt}=
        \Sigma_F\phi_T\frac{\gamma_I\lambda_I}{\lambda_X+\sigma_A\phi_T-\lambda_I}e^{-\lambda_I t}
        -\Sigma_F\phi_T(\frac{(\lambda_X+\sigma_A\phi_T)\gamma_I}{\lambda_X+\sigma_A\phi_T-\lambda_I}
        -(\gamma_I+\gamma_X))e^{-(\lambda_X+\sigma_A\phi_T)t}
\end{equation}

\subsection{Iodine}
\begin{equation}
        \gamma_I\Sigma_F\phi_T e^{-\lambda_I t} = 0
\end{equation}

Here, the constants drop out - 
\begin{equation}
        e^{-\lambda_I t} = 0
\end{equation}

\subsection{Xenon}
\begin{equation} 
        \Sigma_F\phi_T\frac{\gamma_I\lambda_I}{\lambda_X+\sigma_A\phi_T-\lambda_I}e^{-\lambda_I t}
        -\Sigma_F\phi_T(\frac{(\lambda_X+\sigma_A\phi_T)\gamma_I}{\lambda_X+\sigma_A\phi_T-\lambda_I}
        -(\gamma_I+\gamma_X))e^{-(\lambda_X+\sigma_A\phi_T)t}
        = 0
\end{equation}

With this equation, $\Sigma_F\phi_T$ drops out - 
\begin{equation} 
        \frac{\gamma_I\lambda_I}{\lambda_X+\sigma_A\phi_T-\lambda_I}e^{-\lambda_I t}
        -(\frac{(\lambda_X+\sigma_A\phi_T)\gamma_I}{\lambda_X+\sigma_A\phi_T-\lambda_I}
        -(\gamma_I+\gamma_X))e^{-(\lambda_X+\sigma_A\phi_T)t}
        = 0
\end{equation}

\subsection{Solution}
However, neither equation for iodine or xenon cannot be solved explicitly for $t$ because $\ln[0]$ is not defined. There are a variety of techniques that can be applied to obtain the equilibrium time. 

\newpage 

\section{Approximation methods}
\subsection{Newton's method}

For $f(t) = 0$ - 
\begin{equation*}
    t_{n+1} = t_{n} - \frac{f(t_{n})}{f'(t_{n})}
\end{equation*}

\subsubsection{Iodine}
\begin{equation}
        e^{-\lambda_I t} = 0
\end{equation}

\begin{equation}
    f_I(t)=e^{-\lambda_I t}
\end{equation}

\begin{equation}
    f_I'(t)=-\lambda_I e^{-\lambda_I t}
\end{equation}

\subsubsection{Xenon}
\begin{equation} 
        \frac{\gamma_I\lambda_I}{\lambda_X+\sigma_A\phi_T-\lambda_I}e^{-\lambda_I t}
        -(\frac{(\lambda_X+\sigma_A\phi_T)\gamma_I}{\lambda_X+\sigma_A\phi_T-\lambda_I}
        -(\gamma_I+\gamma_X))e^{-(\lambda_X+\sigma_A\phi_T)t}
        = 0
\end{equation}

\begin{equation} 
        f_X(t) =
        \frac{\gamma_I\lambda_I}{\lambda_X+\sigma_A\phi_T-\lambda_I}e^{-\lambda_I t}
        -(\frac{(\lambda_X+\sigma_A\phi_T)\gamma_I}{\lambda_X+\sigma_A\phi_T-\lambda_I}
        -(\gamma_I+\gamma_X))e^{-(\lambda_X+\sigma_A\phi_T)t}
\end{equation}

\begin{equation} 
        f_X'(t) =
        -\frac{\gamma_I\lambda_I\lambda_I}{\lambda_X+\sigma_A\phi_T-\lambda_I}e^{-\lambda_I t}
        +(\lambda_X+\sigma_A\phi_T)(\frac{(\lambda_X+\sigma_A\phi_T)\gamma_I}{\lambda_X+\sigma_A\phi_T-\lambda_I}
        -(\gamma_I+\gamma_X))e^{-(\lambda_X+\sigma_A\phi_T)t}
\end{equation}

\begin{equation} 
        f_X'(t) =
        -\frac{\gamma_I\lambda_I\lambda_I}{\lambda_X+\sigma_A\phi_T-\lambda_I}e^{-\lambda_I t}
        +(\lambda_X+\sigma_A\phi_T)(\frac{(\lambda_X+\sigma_A\phi_T)\gamma_I}{\lambda_X+\sigma_A\phi_T-\lambda_I}
        -\gamma_I-\gamma_X)e^{-(\lambda_X+\sigma_A\phi_T)t}
\end{equation}

\begin{equation} 
        f_X'(t) =
        -\frac{\gamma_I\lambda_I\lambda_I}{\lambda_X+\sigma_A\phi_T-\lambda_I}e^{-\lambda_I t}
        +(\lambda_X+\sigma_A\phi_T)(\frac{\gamma_I(\lambda_X+\sigma_A\phi_T)}{\lambda_X+\sigma_A\phi_T-\lambda_I}
        -\frac{\gamma_I(\lambda_X+\sigma_A\phi_T-\lambda_I)}{\lambda_X+\sigma_A\phi_T-\lambda_I}-\gamma_X)e^{-(\lambda_X+\sigma_A\phi_T)t}
\end{equation}

\begin{equation} 
        f_X'(t) =
        -\frac{\gamma_I\lambda_I\lambda_I}{\lambda_X+\sigma_A\phi_T-\lambda_I}e^{-\lambda_I t}
        +(\frac{\gamma_I(\lambda_X+\sigma_A\phi_T)(\lambda_X+\sigma_A\phi_T)}{\lambda_X+\sigma_A\phi_T-\lambda_I}
        -\frac{\gamma_I(\lambda_X+\sigma_A\phi_T)(\lambda_X+\sigma_A\phi_T-\lambda_I)}{\lambda_X+\sigma_A\phi_T-\lambda_I}
        -\gamma_X(\lambda_X+\sigma_A\phi_T))e^{-(\lambda_X+\sigma_A\phi_T)t}
\end{equation}

\begin{equation} 
        f_X'(t) =
        -\frac{\gamma_I\lambda_I\lambda_I}{\lambda_X+\sigma_A\phi_T-\lambda_I}e^{-\lambda_I t}
        +(\frac{\gamma_I(\lambda_X+\sigma_A\phi_T)(\lambda_X+\sigma_A\phi_T) - 
        \gamma_I(\lambda_X+\sigma_A\phi_T)(\lambda_X+\sigma_A\phi_T-\lambda_I)}{\lambda_X+\sigma_A\phi_T-\lambda_I}
        -\gamma_X(\lambda_X+\sigma_A\phi_T))e^{-(\lambda_X+\sigma_A\phi_T)t}
\end{equation}

\begin{equation} 
        f_X'(t) =
        -\frac{\gamma_I\lambda_I\lambda_I}{\lambda_X+\sigma_A\phi_T-\lambda_I}e^{-\lambda_I t}
        +(\frac{\gamma_I(\lambda_X+\sigma_A\phi_T)(\lambda_X+\sigma_A\phi_T 
        -\lambda_X-\sigma_A\phi_T+\lambda_I)}{\lambda_X+\sigma_A\phi_T-\lambda_I}
        -\gamma_X(\lambda_X+\sigma_A\phi_T))e^{-(\lambda_X+\sigma_A\phi_T)t}
\end{equation}

\begin{equation} 
        f_X'(t) =
        -\frac{\gamma_I\lambda_I\lambda_I}{\lambda_X+\sigma_A\phi_T-\lambda_I}e^{-\lambda_I t}
        +(\frac{\gamma_I\lambda_I(\lambda_X+\sigma_A\phi_T)}{\lambda_X+\sigma_A\phi_T-\lambda_I}
        -\gamma_X(\lambda_X+\sigma_A\phi_T))e^{-(\lambda_X+\sigma_A\phi_T)t}
\end{equation}

\newpage

\section{Steady state concentrations}
The concentration of $I$ and $Xe$ at steady state can be obtained by placing the derivatives equal to zero; $\frac{dI}{dt} = 0$ and $\frac{dX}{dt} = 0$.

Iodine - 
\begin{equation} \label{eq-i-dq}
        \frac{dI}{dt}=\gamma_I \Sigma_F \phi_T-\lambda_I I
\end{equation}

\begin{equation} \label{eq-i-dq}
    \begin{gathered}
        \gamma_I \Sigma_F \phi_T-\lambda_I I = 0 \\
        \lambda_I I = \gamma_I \Sigma_F \phi_T \\
        I_{\infty} = \frac{\gamma_I \Sigma_F \phi_T}{\lambda_I}
    \end{gathered}
\end{equation}


\begin{equation} \label{eq-xe-dq}
        \frac{dX}{dt}=\lambda_I I + \gamma_X \Sigma_F \phi_T - \lambda_X X - \sigma_A \phi_T X \\
\end{equation}

\begin{equation} \label{eq-xe-dq}
    \begin{gathered}
        \lambda_I I + \gamma_X \Sigma_F \phi_T - \lambda_X X - \sigma_A \phi_T X = 0 \\
        \lambda_I (\frac{\gamma_I \Sigma_F \phi_T}{\lambda_I})+\gamma_X \Sigma_F \phi_T-(\lambda_X + \sigma_A\phi_T) X = 0 \\
        (\lambda_X + \sigma_A\phi_T) X = \lambda_I (\frac{\gamma_I \Sigma_F \phi_T}{\lambda_I})+\gamma_X \Sigma_F \phi_T \\
        (\lambda_X + \sigma_A\phi_T) X = \gamma_I \Sigma_F \phi_T+\gamma_X \Sigma_F \phi_T \\
        X_{\infty} = \frac{(\gamma_I+\gamma_X)\Sigma_F\phi_T}{\lambda_X + \sigma_A\phi_T}
    \end{gathered}
\end{equation}



\end{document}
