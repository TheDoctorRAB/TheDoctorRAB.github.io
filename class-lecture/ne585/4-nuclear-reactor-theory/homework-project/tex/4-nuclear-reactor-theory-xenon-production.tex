\documentclass[11pt,a4paper]{article}
\usepackage[landscape,lmargin=1in,rmargin=1in,tmargin=1in,bmargin=1in]{geometry}
\usepackage[pagewise]{lineno} %line numbering
\usepackage{setspace}
\usepackage{ulem} %strikethrough - do not \sout{\cite{}}
\usepackage{xcolor} %change font color
\usepackage{graphicx}
\usepackage{filecontents}
\usepackage{tablefootnote}
\usepackage{footnotehyper}
\usepackage{subfig}
\usepackage[yyyymmdd]{datetime} %date format
\renewcommand{\dateseparator}{.}
\graphicspath{{../img/}} %path to graphics
\setcounter{secnumdepth}{5} %set subsection to nth level
\usepackage{caption}
\captionsetup[table]{skip=11pt} %sets a space after table caption
\usepackage{times}
\usepackage{tabto} %general tabbed spacing
\usepackage{longtable} %need to put label at top under caption then \\ - use spacing
\usepackage[stable,hang,flushmargin]{footmisc} %footnotes in section titles and no indent
\usepackage[round]{natbib} %parenthesis instead of brackets for inline citations
\usepackage{enumitem}
\usepackage{boldline}
\usepackage{makecell}
\usepackage{booktabs}
\usepackage{amssymb}
\usepackage{amsmath}
\usepackage{physics}
\usepackage{tabularx}
\usepackage{multirow}
\usepackage{lscape}
\usepackage{array}
\usepackage{caption}
\usepackage[labelfont=bf]{caption}
\usepackage{chngcntr}
\usepackage{hyperref}

\newcommand{\edit}[1]{\textcolor{blue}{#1}} %shortcut for changing font color on revised text
\newcommand{\fn}[1]{\footnote{#1}} %shortcut for footnote tag
\newcommand*\sq{\mathbin{\vcenter{\hbox{\rule{.3ex}{.3ex}}}}} %makes a small square as a separator $\sq$
\renewcommand\labelenumi{(\theenumi)} %changes 1. to (1) in enumerated list

\usepackage{fancyhdr}
\pagestyle{fancy}
\fancyhf{} %move page number to bottom right
\renewcommand{\headrulewidth}{0.5pt} %turn off line in header
\lhead{\scriptsize }
\chead{\scriptsize Xenon neutron poison production}
\rhead{\scriptsize }
\rfoot{\thepage}

\begin{document}

\begin{titlepage}
    \title{
        NE504 - Nuclear fuel cycle analysis\\
        Xenon neutron poison production\\
    }
    \author{
        R. A. Borrelli
        \\ \\ \\
        University of Idaho $\sq$ Idaho Falls Center for Higher Education
        \\ \\
        Nuclear Engineering and Industrial Management Department
        \\ \\ \\
        r.angelo.borrelli@gmail.com
    }
\clearpage %not have page number on title page
\maketitle
\thispagestyle{empty} %start with page number 1 on second page
\end{titlepage}

\section{Mathematical models}
\subsection{Iodine}
\noindent Rate of change of iodine - 
\begin{equation} \label{eq-i-dq}
    \begin{gathered}
        \frac{dI}{dt}=\gamma_I \Sigma_F \phi_T-\lambda_I I \\
        I(0)=0
    \end{gathered}
\end{equation}

\subsection{Xenon}
\noindent Rate of change of xenon - 
\begin{equation} \label{eq-xe-dq}
    \begin{gathered}
        \frac{dX}{dt}=\lambda_I I + \gamma_X \Sigma_F \phi_T - \lambda_X X - \sigma_A \phi_T X \\
        X(0)=0
    \end{gathered}
\end{equation}

\section{Solutions}
\subsection{Iodine}
\noindent Apply laplace transform to eq. \ref{eq-i-dq} - 
\begin{equation} \label{eq-i-laplace-transform}
    \begin{split}
        & \frac{dI}{dt}=\gamma_I \Sigma_F \phi_T-\lambda_I I \\
        & s\tilde{I}=\frac{1}{s}\gamma_I\Sigma_F\phi_T-\lambda_I\tilde{I}
    \end{split}
\end{equation}
\vspace{\baselineskip}

\noindent Rearrange terms to obtain laplace solution - 
\begin{equation} \label{eq-i-laplace-solution}
    \tilde{I}=\frac{1}{s(s+\lambda_I)}\gamma_I\Sigma_F\phi_T
\end{equation}
\vspace{\baselineskip}

\noindent Invert eq. \ref{eq-i-laplace-solution} to obtain real time domain solution - 
\begin{equation} \label{eq-i-solution}
    I(t)=\frac{\gamma_I\Sigma_F\phi_T}{\lambda_I}(1-e^{-\lambda_I t})
\end{equation}

\subsection{Xenon}
\noindent Substitute the iodine solution in eq. \ref{eq-i-solution} into eq. \ref{eq-xe-dq} - 
\begin{equation} 
    \begin{split}
        & \frac{dX}{dt}=\lambda_I I + \gamma_X \Sigma_F \phi_T - \lambda_X X - \sigma_A \phi_T X \\
        & \frac{dX}{dt}=\lambda_I \frac{\gamma_I\Sigma_F\phi_T}{\lambda_I}(1-e^{-\lambda_I t}) + \gamma_X \Sigma_F \phi_T - \lambda_X X - \sigma_A \phi_T X
    \end{split}
\end{equation}
\vspace{\baselineskip}

\noindent Rearrange the terms and simplify - 
\begin{equation} 
    \begin{split}
        & \frac{dX}{dt}=\gamma_I\Sigma_F\phi_T(1-e^{-\lambda_I t}) + \gamma_X \Sigma_F \phi_T - \lambda_X X - \sigma_A \phi_T X \\
        & \frac{dX}{dt}=\gamma_I\Sigma_F\phi_T - \gamma_I\Sigma_F\phi_T e^{-\lambda_I t} + \gamma_X \Sigma_F \phi_T - \lambda_X X - \sigma_A \phi_T X 
    \end{split}
\end{equation}
\vspace{\baselineskip}

\begin{equation} \label{eq-xe-dq-full}
    \frac{dX}{dt}=\gamma_I\Sigma_F\phi_T - \gamma_I\Sigma_F\phi_T e^{-\lambda_I t} + \gamma_X \Sigma_F \phi_T - (\lambda_X + \sigma_A \phi_T) X
\end{equation}
\vspace{\baselineskip}

\noindent Apply laplace transform to eq. \ref{eq-xe-dq-full} -
\begin{equation} \label{eq-xe-laplace-transform}
    s\tilde{X}=\frac{1}{s}\gamma_I\Sigma_F\phi_T - \frac{1}{s+\lambda_I}\gamma_I\Sigma_F\phi_T + \frac{1}{s}\gamma_X\Sigma_F\phi_T - (\lambda_X+\sigma_A\phi_T)\tilde{X}
\end{equation}
\vspace{\baselineskip}

\noindent Rearrange the terms in eq. \ref{eq-xe-laplace-transform} to obtain the laplace solution - 
\begin{equation}
    (s+[\lambda_X+\sigma_A\phi_T])\tilde{X}=\frac{1}{s}\gamma_I\Sigma_F\phi_T - \frac{1}{s+\lambda_I}\gamma_I\Sigma_F\phi_T + \frac{1}{s}\gamma_X\Sigma_F\phi_T
\end{equation}
\vspace{\baselineskip}

\noindent Then - 
\begin{equation} \label{eq-xe-laplace-solution} 
        \tilde{X}=
        +\frac{1}{s(s+[\lambda_X+\sigma_A\phi_T])}\gamma_I\Sigma_F\phi_T
        -\frac{1}{(s+\lambda_I)(s+[\lambda_X+\sigma_A\phi_T])}\gamma_I\Sigma_F\phi_T 
        +\frac{1}{s(s+[\lambda_X+\sigma_A\phi_T])}\gamma_X\Sigma_F\phi_T
\end{equation}
\vspace{\baselineskip}

\noindent Invert eq. \ref{eq-xe-laplace-solution} to obtain real time domain solution - 
\begin{equation} \label{eq-xe-solution} 
        X(t)=
        +\frac{\gamma_I\Sigma_F\phi_T}{\lambda_X+\sigma_A\phi_T}(1-e^{-(\lambda_X+\sigma_A\phi_T)t}) 
        -\frac{\gamma_I\Sigma_F\phi_T}{\lambda_X+\sigma_A\phi_T-\lambda_I}(e^{-\lambda_I t}-e^{-(\lambda_X+\sigma_A\phi_T)t}) 
        +\frac{\gamma_X\Sigma_F\phi_T}{\lambda_X+\sigma_A\phi_T}(1-e^{-(\lambda_X+\sigma_A\phi_T)t}) 
\end{equation}

\section{Equilibrium time}
\noindent The equilibrium time for xenon should be when the concentration achieves a steady state; i.e., the rate of change in the concentration is zero. \\

\noindent Then, compute the derivative of the xenon concentration model in eq. \ref{eq-xe-solution} - 
\begin{equation} \label{eq-xe-derivative}
    \begin{aligned}
        \frac{dX}{dt}=\\
        & +[\frac{\gamma_I\Sigma_F\phi_T}{\lambda_X+\sigma_A\phi_T}]((\lambda_X+\sigma_A\phi_T)e^{-(\lambda_X+\sigma_A\phi_T)t}) \\
        & -[\frac{\gamma_I\Sigma_F\phi_T}{\lambda_X+\sigma_A\phi_T-\lambda_I}]((\lambda_X+\sigma_A\phi_T)e^{-(\lambda_X+\sigma_A\phi_T)t}-\lambda_Ie^{-\lambda_I t}) \\ 
        & +[\frac{\gamma_X\Sigma_F\phi_T}{\lambda_X+\sigma_A\phi_T}]((\lambda_X+\sigma_A\phi_T)e^{-(\lambda_X+\sigma_A\phi_T)t})
    \end{aligned}
\end{equation}
\vspace{\baselineskip}

\noindent Multiply out all the terms - 
\begin{equation}
    \begin{aligned}
        \frac{dX}{dt}=\\
        & +\gamma_I\Sigma_F\phi_T e^{-(\lambda_X+\sigma_A\phi_T)t} \\
        & -\frac{(\gamma_I\Sigma_F\phi_T)(\lambda_X+\sigma_A\phi_T)}{(\lambda_X+\sigma_A\phi_T-\lambda_I)}e^{-(\lambda_X+\sigma_A\phi_T)t} \\
        & +\frac{(\gamma_I\Sigma_F\phi_T)(\lambda_I)}{(\lambda_X+\sigma_A\phi_T-\lambda_I)}e^{-\lambda_I t} \\
        & +\gamma_X\Sigma_F\phi_T e^{-(\lambda_X+\sigma_A\phi_T)t}
    \end{aligned}
\end{equation}
\vspace{\baselineskip}

\noindent Let $\frac{dX}{dt} = 0$ and group the terms by exponential -
\begin{equation}
    (\gamma_I\Sigma_F\phi_T+\gamma_X\Sigma_F\phi_T-\frac{(\gamma_I\Sigma_F\phi_T)(\lambda_X+\sigma_A\phi_T)}{(\lambda_X+\sigma_A\phi_T-\lambda_I)})e^{-(\lambda_X+\sigma_A\phi_T)t}+\frac{(\gamma_I\Sigma_F\phi_T)(\lambda_I)}{(\lambda_X+\sigma_A\phi_T-\lambda_I)}e^{-\lambda_I t}=0
\end{equation}
\vspace{\baselineskip}

\noindent The next series of steps is just some algebra - 
\begin{equation}
    \begin{split}
        & -(\gamma_I\Sigma_F\phi_T+\gamma_X\Sigma_F\phi_T-\frac{(\gamma_I\Sigma_F\phi_T)(\lambda_X+\sigma_A\phi_T)}{(\lambda_X+\sigma_A\phi_T-\lambda_I)})e^{-(\lambda_X+\sigma_A\phi_T)t}=\frac{(\gamma_I\Sigma_F\phi_T)(\lambda_I)}{(\lambda_X+\sigma_A\phi_T-\lambda_I)}e^{-\lambda_I t} \\
        & (-\gamma_I\Sigma_F\phi_T-\gamma_X\Sigma_F\phi_T+\frac{(\gamma_I\Sigma_F\phi_T)(\lambda_X+\sigma_A\phi_T)}{(\lambda_X+\sigma_A\phi_T-\lambda_I)})e^{-(\lambda_X+\sigma_A\phi_T)t}=\frac{(\gamma_I\Sigma_F\phi_T)(\lambda_I)}{(\lambda_X+\sigma_A\phi_T-\lambda_I)}e^{-\lambda_I t}
    \end{split}
\end{equation}
\vspace{\baselineskip}

\begin{equation} \label{eq-xe-derivative-setup}
        (\frac{(\gamma_I\Sigma_F\phi_T)(\lambda_X+\sigma_A\phi_T)}{(\lambda_X+\sigma_A\phi_T-\lambda_I)}-\gamma_I\Sigma_F\phi_T-\gamma_X\Sigma_F\phi_T)e^{-(\lambda_X+\sigma_A\phi_T)t}=\frac{(\gamma_I\Sigma_F\phi_T)(\lambda_I)}{(\lambda_X+\sigma_A\phi_T-\lambda_I)}e^{-\lambda_I t}
\end{equation}
\vspace{\baselineskip}

\noindent Divide through eq. \ref{eq-xe-derivative-setup} by $e^{-\lambda_I t}$ and $(\frac{(\gamma_I\Sigma_F\phi_T)(\lambda_X+\sigma_A\phi_T)}{(\lambda_X+\sigma_A\phi_T-\lambda_I)}-\gamma_I\Sigma_F\phi_T-\gamma_X\Sigma_F\phi_T)$ -
\begin{equation} 
    \frac{e^{-(\lambda_X+\sigma_A\phi_T)t}}{e^{-\lambda_I t}}=\frac{\frac{(\gamma_I\Sigma_F\phi_T)(\lambda_I)}{(\lambda_X+\sigma_A\phi_T-\lambda_I)}}{(\frac{(\gamma_I\Sigma_F\phi_T)(\lambda_X+\sigma_A\phi_T)}{(\lambda_X+\sigma_A\phi_T-\lambda_I)}-\gamma_I\Sigma_F\phi_T-\gamma_X\Sigma_F\phi_T)}
\end{equation}
\vspace{\baselineskip}

\noindent Compute the exponential division - 
\begin{equation}
    e^{(\lambda_I-(\lambda_X+\sigma_A\phi_T))t}=\frac{\frac{(\gamma_I\Sigma_F\phi_T)(\lambda_I)}{(\lambda_X+\sigma_A\phi_T-\lambda_I)}}{(\frac{(\gamma_I\Sigma_F\phi_T)(\lambda_X+\sigma_A\phi_T)}{(\lambda_X+\sigma_A\phi_T-\lambda_I)}-\gamma_I\Sigma_F\phi_T-\gamma_X\Sigma_F\phi_T)}
\end{equation}
\vspace{\baselineskip}

\noindent Compute the natural logarithm - 
\begin{equation}
    (\lambda_I-(\lambda_X+\sigma_A\phi_T))t=ln[\frac{\frac{(\gamma_I\Sigma_F\phi_T)(\lambda_I)}{(\lambda_X+\sigma_A\phi_T-\lambda_I)}}{(\frac{(\gamma_I\Sigma_F\phi_T)(\lambda_X+\sigma_A\phi_T)}{(\lambda_X+\sigma_A\phi_T-\lambda_I)}-\gamma_I\Sigma_F\phi_T-\gamma_X\Sigma_F\phi_T)}]
\end{equation}
\vspace{\baselineskip}

\noindent Solve for time ($t$) - 
\begin{equation} \label{eq-xe-equil-time}
    t=\frac{1}{\lambda_I-(\lambda_X+\sigma_A\phi_T)}\cdot  ln[\frac{\frac{(\gamma_I\Sigma_F\phi_T)(\lambda_I)}{(\lambda_X+\sigma_A\phi_T-\lambda_I)}}{(\frac{(\gamma_I\Sigma_F\phi_T)(\lambda_X+\sigma_A\phi_T)}{(\lambda_X+\sigma_A\phi_T-\lambda_I)}-\gamma_I\Sigma_F\phi_T-\gamma_X\Sigma_F\phi_T)}]
\end{equation}
\vspace{\baselineskip}

\noindent Simplify $\ln[\cdot]$ - 
\begin{equation}
    \frac{\frac{(\gamma_I\Sigma_F\phi_T)(\lambda_I)}{(\lambda_X+\sigma_A\phi_T-\lambda_I)} \cdot (\lambda_X+\sigma_A\phi_T-\lambda_I)}{(\frac{(\gamma_I\Sigma_F\phi_T)(\lambda_X+\sigma_A\phi_T)}{(\lambda_X+\sigma_A\phi_T-\lambda_I)} \cdot (\lambda_X+\sigma_A\phi_T-\lambda_I) - \gamma_I\Sigma_F\phi_T(\lambda_X+\sigma_A\phi_T-\lambda_I) - \gamma_X\Sigma_F\phi_T(\lambda_X+\sigma_A\phi_T-\lambda_I))}
\end{equation}
\vspace{\baselineskip}

\begin{equation}
    \frac{
        \gamma_I\lambda_I\Sigma_F\phi_T
    }
    {
        (\gamma_I\Sigma_F\phi_T)(\lambda_X+\sigma_A\phi_T)
        - \gamma_I\Sigma_F\phi_T(\lambda_X+\sigma_A\phi_T-\lambda_I) 
        - \gamma_X\Sigma_F\phi_T(\lambda_X+\sigma_A\phi_T-\lambda_I)
    }
\end{equation}
\vspace{\baselineskip}

\begin{equation}
    \frac{
        \gamma_I\lambda_I\Sigma_F\phi_T
    }
    {
        (\gamma_I\Sigma_F\phi_T)(\lambda_X+\sigma_A\phi_T)
        - (\gamma_I\Sigma_F\phi_T)(\lambda_X+\sigma_A\phi_T-\lambda_I) 
        - (\gamma_X\Sigma_F\phi_T)(\lambda_X+\sigma_A\phi_T-\lambda_I)
    }
\end{equation}
\vspace{\baselineskip}

\begin{equation}
    \frac{
        \gamma_I\lambda_I\Sigma_F\phi_T
    }
    {
        (\gamma_I\Sigma_F\phi_T)((\lambda_X+\sigma_A\phi_T) - (\lambda_X+\sigma_A\phi_T-\lambda_I))
        - (\gamma_X\Sigma_F\phi_T)(\lambda_X+\sigma_A\phi_T-\lambda_I)
    }
\end{equation}
\vspace{\baselineskip}

\begin{equation}
    \frac{
        \gamma_I\lambda_I\Sigma_F\phi_T
    }
    {
        (\gamma_I\Sigma_F\phi_T)(\gamma_X + \sigma_A\phi_T - \gamma_X - \sigma_A\phi_T + \lambda_I)
        - (\gamma_X\Sigma_F\phi_T)(\lambda_X+\sigma_A\phi_T-\lambda_I)
    }
\end{equation}
\vspace{\baselineskip}

\begin{equation}
    \frac{
        \gamma_I\lambda_I\Sigma_F\phi_T
    }
    {
        \gamma_I\lambda_I\Sigma_F\phi_T
        - (\gamma_X\Sigma_F\phi_T)(\lambda_X+\sigma_A\phi_T-\lambda_I)
    }
\end{equation}
\vspace{\baselineskip}

\begin{equation}
    \frac{
        \gamma_I\lambda_I\Sigma_F(\phi_T)
    }
    {
        \gamma_I\lambda_I\Sigma_F(\phi_T)
        - (\gamma_X\Sigma_F(\phi_T))(\lambda_X+\sigma_A\phi_T-\lambda_I)
    }
\end{equation}
\vspace{\baselineskip}

\begin{equation} \label{eq-ln}
    \frac{
        \gamma_I\lambda_I\Sigma_F
    }
    {
        \gamma_I\lambda_I\Sigma_F
        - \gamma_X\Sigma_F(\lambda_X+\sigma_A\phi_T-\lambda_I)
    }
\end{equation}
\vspace{\baselineskip}
\vspace{\baselineskip}

\noindent Finally, substitute eq. \ref{eq-ln} back into eq. \ref{eq-xe-equil-time} -  
\begin{equation} \label{eq-xe-equil-time}
    t=\frac{1}{\lambda_I-(\lambda_X+\sigma_A\phi_T)} \cdot  
    ln[    \frac{
        \gamma_I\lambda_I\Sigma_F
    }
    {
        \gamma_I\lambda_I\Sigma_F
        - \gamma_X\Sigma_F(\lambda_X+\sigma_A\phi_T-\lambda_I)
    }]
\end{equation}

\end{document}
