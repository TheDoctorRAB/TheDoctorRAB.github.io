\documentclass[11pt,a4paper]{article}
\usepackage[lmargin=1in,rmargin=1in,tmargin=1in,bmargin=1in]{geometry}
\usepackage[pagewise]{lineno} %line numbering
\usepackage{setspace}
\usepackage{ulem} %strikethrough - do not \sout{\cite{}}
\usepackage{xcolor} %change font color
\usepackage{graphicx}
\usepackage{filecontents}
\usepackage{tablefootnote}
\usepackage{footnotehyper}
\usepackage{subfig}
\usepackage[yyyymmdd]{datetime} %date format
\renewcommand{\dateseparator}{.}
\graphicspath{{../img/}} %path to graphics
\setcounter{secnumdepth}{5} %set subsection to nth level
\usepackage{caption}
\captionsetup[table]{skip=11pt} %sets a space after table caption
\usepackage{times}
\usepackage{tabto} %general tabbed spacing
\usepackage{longtable} %need to put label at top under caption then \\ - use spacing
\usepackage[stable,hang,flushmargin]{footmisc} %footnotes in section titles and no indent
\usepackage[round]{natbib} %parenthesis instead of brackets for inline citations
\usepackage{enumitem}
\usepackage{boldline}
\usepackage{makecell}
\usepackage{booktabs}
\usepackage{amssymb}
\usepackage{amsmath}
\usepackage{physics}
\usepackage{tabularx}
\usepackage{multirow}
\usepackage{lscape}
\usepackage{array}
\usepackage{caption}
\usepackage[labelfont=bf]{caption}
\usepackage{chngcntr}
\usepackage{hyperref}

\newcommand{\edit}[1]{\textcolor{blue}{#1}} %shortcut for changing font color on revised text
\newcommand{\fn}[1]{\footnote{#1}} %shortcut for footnote tag
\newcommand*\sq{\mathbin{\vcenter{\hbox{\rule{.3ex}{.3ex}}}}} %makes a small square as a separator $\sq$
\renewcommand\labelenumi{(\theenumi)} %changes 1. to (1) in enumerated list

\usepackage{fancyhdr}
\pagestyle{fancy}
\fancyhf{} %move page number to bottom right
\renewcommand{\headrulewidth}{0.5pt} %turn off line in header
\lhead{\scriptsize }
\chead{\scriptsize Two group diffusion}
\rhead{\scriptsize }
\rfoot{\thepage}

\begin{document}

\begin{titlepage}
    \title{
        NE504 - Nuclear fuel cycle analysis\\
        Two group diffusion\\
    }
    \author{
        R. A. Borrelli
        \\ \\ \\
        University of Idaho $\sq$ Idaho Falls Center for Higher Education
        \\ \\
        Nuclear Engineering and Industrial Management Department
        \\ \\ \\
        r.angelo.borrelli@gmail.com
    }
\clearpage %not have page number on title page
\maketitle
\thispagestyle{empty} %start with page number 1 on second page
\end{titlepage}

\section{Equation of continuity}
\noindent The equation of continuity for neutrons describes behavior in a reactor.\\

\noindent [rate of change of neutrons] = [production rate] - [absorption rate] - [leakage rate]\\

\begin{equation} \label{eq-1-continuity}
    \frac{d}{dt}\int_V ndV=\int_V sdV-\int_V\Sigma_A\phi dV-\int_V\underline{J}\cdot\underline{n}dA
\end{equation}

\noindent The terms are - 
\begin{itemize}[leftmargin=*,topsep=0pt]
    \item $\int_V ndV$ - total number of neutrons
    \item $\frac{d}{dt} \int_V ndV$ - rate of change
    \item $\int_V sdV$ - production rate
    \item $\int_V \Sigma_A \phi dV$ - absorption rate
    \item $\int_A \underline{J} \cdot \underline{n} dA$ leakage rate
\end{itemize}
\vspace{\baselineskip}

\noindent Applying some mathematical relationships - 
\begin{equation}
    \begin{split}
        & \frac{d}{dt} \int_V ndV = \int_V \frac{\partial n}{\partial t} dV \\
        & \int_A \underline{J} \cdot \underline{n} dA = \int_V \nabla \underline{J} dV
    \end{split}
\end{equation}

\noindent Substitute into eq. \ref{eq-1-continuity} - 
\begin{equation}
    \int_V \frac{\partial{n}}{\partial{t}} dV = \int_V s dV - \int_V \Sigma_A\phi dV - \int_V \nabla \underline{J} dV
\end{equation}

\noindent Because the control volume is the same - 
\begin{equation} \label{eq-continuity}
    \frac{\partial{n}}{\partial{t}}=s-\Sigma_A\phi-\nabla{J}
\end{equation}

\noindent Apply Fick's law - 
\begin{equation}
    \begin{split}
        & \frac{\partial{n}}{\partial{t}}=D\nabla^2{\phi}-\Sigma_A\phi+s \\
        & \frac{1}{v}\frac{\partial{\phi}}{\partial{t}}=D\nabla^2{\phi}-\Sigma_A\phi+s
    \end{split}
\end{equation}

\noindent Assume steady state - 
\begin{equation} \label{eq-general-diffusion}
    D\nabla^2{\phi}-\Sigma_A\phi+s=0
\end{equation}

\section{Multigroup diffusion theory}
\noindent The general multigroup diffusion equation based on eq. \ref{eq-general-diffusion} for an arbitrary group $g$ is - 
\begin{equation} \label{eq-multigroup-diffusion}
    D_g\nabla^2{\phi_g}-\Sigma_A^g\phi_g-\sum_{h=g+1}^N{\Sigma_{g \rightarrow h}\phi_g}+\sum_{h=1}^{g-1}{\Sigma_{h \rightarrow g}\phi_h}+s_g=0
\end{equation}

\noindent In addition to absorption in group $g$, there is the loss due to scatter from group $g$ to $h$ but also a source terms due to scatter from $h$ to $g$.

\noindent For 2 groups, then; i.e., a fast and thermal group, eq. \ref{eq-multigroup-diffusion} is - 
\begin{equation}
    \begin{split}
        & D_1\nabla^2{\phi_1}-\Sigma_A^1\phi_1-\Sigma_{1 \rightarrow 2}\phi_1+\Sigma_{2 \rightarrow 1}\phi_2+s=0 \\
        & D_2\nabla^2{\phi_2}-\Sigma_A^2\phi_2-\Sigma_{2 \rightarrow 1}\phi_2+\Sigma_{1 \rightarrow 2}\phi_1=0
    \end{split}
\end{equation}

\noindent Neutrons will not scatter from the thermal group ($2$) to the fast group ($1$) - 
\begin{equation}
    \begin{split}
        & D_1\nabla^2{\phi_1}-\Sigma_A^1\phi_1-\Sigma_{1 \rightarrow 2}\phi_1+s=0 \\
        & D_2\nabla^2{\phi_2}-\Sigma_A^2\phi_2+\Sigma_{1 \rightarrow 2}\phi_1=0
    \end{split}
\end{equation}

\section{Solution}
\begin{enumerate}[leftmargin=*,topsep=0pt]
    \item Assume a one-dimensional, infinite slab.
    \item Compute the derivatives - 
        \begin{equation} \label{eq-fast-diffusion}
            \frac{d^2\phi_1}{dx^2}-\frac{1}{K^2}\phi_1+d=0 
        \end{equation}
        \begin{equation} \label{eq-thermal-diffusion}
            \frac{d^2\phi_2}{dx^2}-\frac{1}{L^2}\phi_2+\Sigma_{1 \rightarrow 2}\phi_1=0
        \end{equation}
    \item[] Where - 
        \begin{equation*}
            \begin{split}
                & L^2=\frac{D_2}{\Sigma_A^2} \\
                & K^2=\frac{D_1}{\Sigma_A^1+\Sigma_{1\rightarrow 2}} \\
                & d=\frac{s}{D_1}
            \end{split}
        \end{equation*}
    \item[] Let - 
        \begin{equation*}
            \frac{d^2\phi}{dx^2} \equiv \phi''
        \end{equation*}
\end{enumerate}

\subsection{Fast group}
\begin{enumerate}[leftmargin=*,topsep=0pt]
    \item Assume the solution is the sum of a homogeneous and a particular solution.
        \begin{equation}
            \phi_1(x)=\phi_H(x)+\phi_P(x)
        \end{equation}
    \item Making use of eq. \ref{eq-fast-diffusion} -  
        \begin{equation}
            \frac{d^2\phi_H}{dx^2}-\frac{1}{K^2}\phi_H=0
        \end{equation}
    \item Assume the homogeneous solution -
        \begin{equation} \label{eq-fast-diffusion-homogeneous-solution}
            Ae^{-\frac{x}{K}}+Ce^{\frac{x}{K}}
        \end{equation}
    \item Assume the particular solution is an arbitrary constant - 
        \begin{equation} \label{eq-fast-diffusion-general-particular-solution}
            \phi_P=B
        \end{equation}
    \item Substitute eq. \ref{eq-fast-diffusion-general-particular-solution} in to eq. \ref{eq-fast-diffusion} - 
        \begin{equation}
            0-\frac{1}{K^2}B+d=0
        \end{equation}
    \item The particular solution is then - 
        \begin{equation} \label{eq-fast-diffusion-particular-solution}
            \phi_P=dK^2
        \end{equation}
    \item The fast flux is then -
        \begin{equation} \label{eq-fast-diffusion-solution}
            \phi_1=A_1e^{-\frac{x}{K}}+C_1e^{\frac{x}{K}}+dK^2
        \end{equation}
\end{enumerate}

\subsection{Thermal flux}
\begin{enumerate}[leftmargin=*,topsep=0pt]
    \item Substitute eq. \ref{eq-fast-diffusion-solution} into eq. \ref{eq-thermal-diffusion}, using the shorthand notation ($''$) for the second derivative - 
        \begin{equation} \label{eq-thermal-diffusion-model}
            \phi''_2-\frac{1}{L^2}\phi_2+\Sigma_1 A_1e^{-\frac{x}{K}}+\Sigma_1 C_1e^{\frac{x}{K}}+\Sigma_1 dK^2=0
        \end{equation}
    \item[] Where - 
        \begin{equation*}
            \Sigma_{1 \rightarrow 2} \equiv \Sigma_1
        \end{equation*}
    \item Following the same procedure, assume the solution is the sum of a homogeneous and a particular solution. 
        \begin{equation} \label{eq-thermal-general-solution}
            \phi_2=\phi_H+\phi_P
        \end{equation}
    \item Making use of eq. \ref{eq-thermal-diffusion} - 
        \begin{equation}
            \phi''_H-\frac{1}{L^2}\phi_H=0
        \end{equation}
    \item Assume the homogeneous solution -
        \begin{equation} \label{eq-thermal-homogeneous-solution}
            \phi_H=A_2e^{-\frac{x}{L}}+C_2e^{\frac{x}{L}}
        \end{equation}
    \item Apply the following \href{https://www.math.uh.edu/~etgen/5389-Chap3.pdf}{theorem} for the particular solution - 
        \begin{enumerate}[leftmargin=*,topsep=0pt]
            \item If $z_1(x)$ is a particular solution of - 
                \begin{equation*}
                    y''+p(x)y'+q(x)y=f(x)
                \end{equation*}
            \item and $z_2(x)$ is a particular solution of - 
                \begin{equation*}
                    y''+p(x)y'+q(x)y=g(x)
                \end{equation*}
            \item Then, for any number of forcing functions, $z=z_1(x)+z_2(x)$ is a particular solution of -
                \begin{equation*}
                    y''+p(x)y'+q(x)y=f(x)+g(x)
                \end{equation*}
        \end{enumerate}
    \item Therefore, because there are three forcing functions in eq. \ref{eq-thermal-diffusion-model} - 
        \begin{equation}
            \phi_P=\phi_F+\phi_G+\phi_I
        \end{equation}
    \item Based on the theorem - 
        \begin{equation} \label{eq-thermal-particular}
            \begin{split}
                & \phi''_F-\frac{1}{L^2}\phi_F+\Sigma_1 A_1 e^{-\frac{x}{K}} \\
                & \phi''_G-\frac{1}{L^2}\phi_G+\Sigma_1 C_1 e^{\frac{x}{K}} \\
                & \phi''_I-\frac{1}{L^2}\phi_I+\Sigma_1 dK^2
            \end{split}
        \end{equation}
    \item[] The diffusion length ($L^2$) does not need to be split because the geometry is the same.
    \item Assume the correponding particular solutions for each forcing function - 
        \begin{equation} \label{eq-thermal-assumed-particular}
            \begin{split}
                & \phi_F^P=B_1e^{-\frac{x}{K}} \\
                & \phi_G^P=B_2e^{\frac{x}{K}} \\
                & \phi_I^P=B_3
            \end{split}
        \end{equation}
    \item Compute the second derivative of each particular solution - 
        \begin{equation}
            \begin{split} \label{eq-thermal-assumed-particular-derivative}
                & \phi''_F=\frac{B_1}{K^2}e^{-\frac{x}{K}} \\
                & \phi''_G=\frac{B_2}{K^2}e^{\frac{x}{K}} \\
                & \phi''_H=0
            \end{split}
        \end{equation}
    \item Substitute eq. \ref{eq-thermal-assumed-particular-derivative} and \ref{eq-thermal-assumed-particular} into eq. \ref{eq-thermal-particular} - 
        \begin{equation}
            \begin{split}
                & \frac{B_1}{K^2}e^{-\frac{x}{K}}-\frac{B_1}{L^2}e^{-\frac{x}{K}}+\Sigma_1 A_1 e^{-\frac{x}{K}}=0 \\
                & \frac{B_2}{K^2}e^{\frac{x}{K}}-\frac{B_2}{L^2}e^{\frac{x}{K}}+\Sigma_1 C_1 e^{\frac{x}{K}}=0 \\
                & 0-\frac{B_3}{L^2}+\Sigma_1 dK^2=0
            \end{split}
        \end{equation}
    \item Simplfying - 
        \begin{equation}
            \begin{split}
                & \frac{B_1}{K^2}-\frac{B_1}{L^2}+\Sigma_1 A_1 =0 \\
                & \frac{B_2}{K^2}-\frac{B_2}{L^2}+\Sigma_1 C_1 =0 \\
                & \frac{B_3}{L^2}-\Sigma_1 dK^2 = 0
            \end{split}
        \end{equation}
    \item Solve for the constants - 
        \begin{equation} \label{eq-thermal-particular-constants}
            \begin{split}
                & B_1=\frac{\Sigma_1 A_1 K^2 L^2}{K^2-L^2} \\
                & B_2=\frac{\Sigma_1 C_1 K^2 L^2}{K^2-L^2} \\
                & B_3=\Sigma_1dK^2L^2
            \end{split}
        \end{equation}
    \item Substitute the constants in eq. \ref{eq-thermal-particular-constants} into eq. \ref{eq-thermal-assumed-particular} - 
        \begin{equation} \label{eq-thermal-particular-solution}
            \begin{split}
                & \phi_F^P=\frac{\Sigma_1 A_1 K^2 L^2}{K^2-L^2} e^{-\frac{x}{K}} \\
                & \phi_G^P=\frac{\Sigma_1 C_1 K^2 L^2}{K^2-L^2} e^{\frac{x}{K}} \\
                & \phi_I^P=\Sigma_1 d K^2 L^2
            \end{split}
        \end{equation}
    \item Finally, substitute the particular solutions in eq. \ref{eq-thermal-particular-solution} and the homogeneous solution in eq. \ref{eq-thermal-homogeneous-solution} into eq. \ref{eq-thermal-general-solution} to obtain the termal flux - 
        \begin{equation} \label{eq-thermal-diffusion-solution}
            \phi_2=A_2e^{-\frac{x}{L}}+C_2e^{\frac{x}{L}}+\frac{\Sigma_1 A_1 K^2 L^2}{K^2-L^2} e^{-\frac{x}{K}}+\frac{\Sigma_1 C_1 K^2 L^2}{K^2-L^2} e^{\frac{x}{K}}+\Sigma_1 d K^2 L^2
        \end{equation}
\end{enumerate}

\newpage 

\section{Solution check}
\subsection{Fast flux}
\begin{enumerate}[leftmargin=*,topsep=0pt]
    \item The fast flux is -
        \begin{equation} \label{eq-fast-flux}
            \phi_1=A_1e^{-\frac{x}{K}}+C_1e^{\frac{x}{K}}+dK^2
        \end{equation}
    \item And the solution must satisfy - 
        \begin{equation} \label{eq-fast-flux-check}
            \frac{d^2\phi_1}{dx^2}-\frac{1}{K^2}\phi_1+d=0 
        \end{equation}
    \item Compute the second derivative of eq. \ref{eq-fast-flux} - 
        \begin{equation}
                \phi''_1=\frac{A_1}{K^2}e^{-\frac{x}{K}}+\frac{C_1}{K^2}e^{\frac{x}{K}} \\
        \end{equation}
    \item Substitute the result into eq. \ref{eq-fast-flux-check} and simplify - 
        \begin{equation*}
                (\frac{A_1}{K^2}e^{-\frac{x}{K}}+\frac{C_1}{K^2}e^{\frac{x}{K}})-\frac{1}{K^2}[A_1e^{-\frac{x}{K}}+C_1e^{\frac{x}{K}}+dK^2]+d=0
        \end{equation*}
        \begin{equation*}
                \frac{A_1}{K^2}e^{-\frac{x}{K}}+\frac{C_1}{K^2}e^{\frac{x}{K}}-\frac{A_1}{K^2}e^{-\frac{x}{K}}-\frac{C_1}{K^2}e^{\frac{x}{K}}-d+d=0
        \end{equation*}
    \item[] Check satisfied.
\end{enumerate}

\subsection{Thermal flux}
\begin{enumerate}[leftmargin=*,topsep=0pt]
    \item The thermal flux is - 
        \begin{equation} \label{eq-thermal-flux}
            \phi_2=A_2e^{-\frac{x}{L}}+C_2e^{\frac{x}{L}}+\frac{\Sigma_1 A_1 K^2 L^2}{K^2-L^2} e^{-\frac{x}{K}}+\frac{\Sigma_1 C_1 K^2 L^2}{K^2-L^2} e^{\frac{x}{K}}+\Sigma_1 d K^2 L^2
        \end{equation}
    \item And the solution must satisfy - 
        \begin{equation} \label{eq-thermal-flux-check}
            \frac{d^2\phi_2}{dx^2}-\frac{1}{L^2}\phi_2+\Sigma_1\phi_1=0
        \end{equation}
    \item Compute the second derivative of eq. \ref{eq-thermal-flux} - 
        \begin{equation}
            \phi''_2=\frac{A_2}{L^2}e^{-\frac{x}{L}}+\frac{C_2}{L^2}e^{\frac{x}{L}}+\frac{\Sigma_1 A_1 K^2 L^2}{(K^2-L^2)K^2} e^{-\frac{x}{K}}+\frac{\Sigma_1 C_1 K^2 L^2}{(K^2-L^2)K^2} e^{\frac{x}{K}}
        \end{equation}
    \item Substitute the result into eq. \ref{eq-thermal-flux-check} and simplify - 
        \begin{equation*}
            \begin{aligned}
                0=\\
                & +[\frac{A_2}{L^2}e^{-\frac{x}{L}}+\frac{C_2}{L^2}e^{\frac{x}{L}}+\frac{\Sigma_1 A_1 K^2 L^2}{(K^2-L^2)K^2} e^{-\frac{x}{K}}+\frac{\Sigma_1 C_1 K^2 L^2}{(K^2-L^2)K^2} e^{\frac{x}{K}}] \\
                & -[\frac{A_2}{L^2}e^{-\frac{x}{L}}+\frac{C_2}{L^2}e^{\frac{x}{L}}+\frac{\Sigma_1 A_1 K^2 L^2}{(K^2-L^2)L^2} e^{-\frac{x}{K}}+\frac{\Sigma_1 C_1 K^2 L^2}{(K^2-L^2)L^2} e^{\frac{x}{K}}+\Sigma_1 d K^2] \\
                & +[\Sigma_1A_1e^{-\frac{x}{K}}+\Sigma_1C_1e^{\frac{x}{K}}+\Sigma_1dK^2]
            \end{aligned}
        \end{equation*}
        \begin{equation*}
            \begin{aligned}
                0=\\
                & +(\frac{A_2}{L^2}-\frac{A_2}{L^2})e^{-\frac{x}{L}}+(\frac{C_2}{L^2}-\frac{C_2}{L^2})e^{\frac{x}{L}}+(\Sigma_1dK^2-\Sigma_1dK^2) \\
                & +[\frac{\Sigma_1A_1K^2L^2}{(K^2-L^2)K^2}-\frac{\Sigma_1A_1K^2L^2}{(K^2-L^2)L^2}+\Sigma_1A_1]e^{-\frac{x}{K}} \\
                & +[\frac{\Sigma_1C_1K^2L^2}{(K^2-L^2)K^2}-\frac{\Sigma_1C_1K^2L^2}{(K^2-L^2)L^2}+\Sigma_1C_1]e^{\frac{x}{K}}
            \end{aligned}
        \end{equation*}
        \begin{equation*}
            \begin{aligned}
                0=\\
                & +(\frac{A_2}{L^2}-\frac{A_2}{L^2})e^{-\frac{x}{L}}+(\frac{C_2}{L^2}-\frac{C_2}{L^2})e^{\frac{x}{L}}+(\Sigma_1dK^2-\Sigma_1dK^2) \\
                & +[\frac{\Sigma_1A_1L^2}{K^2-L^2}-\frac{\Sigma_1A_1K^2}{K^2-L^2}+\frac{\Sigma_1A_1(K^2-L^2)}{K^2-L^2}]e^{-\frac{x}{K}} \\
                & +[\frac{\Sigma_1C_1L^2}{K^2-L^2}-\frac{\Sigma_1C_1K^2}{K^2-L^2}+\frac{\Sigma_1C_1(K^2-L^2)}{K^2-L^2}]e^{\frac{x}{K}}
            \end{aligned}
        \end{equation*}
    \item[]Check satisfied.
\end{enumerate}

\end{document}
